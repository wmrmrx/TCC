% Poster get from https://github.com/victorsenam/tcc/blob/master/poster/main.tex

\documentclass[final]{beamer}
\usepackage[size=a1,orientation=portrait,scale=1.3]{beamerposter}

\usepackage[brazil]{babel}
\usepackage[utf8]{inputenc}
\usepackage[T1]{fontenc}
\usepackage{framed,graphicx,xcolor} % for shaded box
\usepackage{mathtools}%
\usepackage{tikz}
\usetikzlibrary{matrix,shapes,positioning,shadows,trees,patterns}

\usepackage[shortlabels]{enumitem}
\usepackage[numbers]{natbib}
\bibliographystyle{plainnat}
  \def\bibfont{\small}

\sloppy

%----------------------------------------------------------------------------------------
%	SHORTCUTS
%----------------------------------------------------------------------------------------
\newcommand{\B}[1]{\mathbb{#1}}
\newcommand{\Cl}[1]{\ensuremath{\mathcal{#1}}}

\newcommand{\sse}{\Leftrightarrow}
\newcommand{\so}{\Rightarrow}
\newcommand{\se}{\Leftarrow}
\newcommand{\rec}{\leftarrow}

\newcommand{\tdots}{\,.\,.\,}

%----------------------------------------------------------------------------------------
%	BEAMER STYLE
%----------------------------------------------------------------------------------------

\usetheme{poster}
\setbeamercolor{block title}{fg=dblue,bg=white}
\setbeamercolor{block body}{fg=black,bg=white}
\setbeamercolor{block alerted title}{fg=dblue,bg=gray!50}
\setbeamercolor{block alerted body}{fg=black,bg=gray!20}
\setbeamercolor{block prob}{fg=black,bg=white}
\setbeamertemplate{caption}[numbered]

%----------------------------------------------------------------------------------------
%	CUSTOM STYLING
%----------------------------------------------------------------------------------------

\newenvironment<>{prob}{
    \begin{beamercolorbox}[sep=1ex,center,dp={1ex}]{block prob}
    \textcolor{dblue}{\textbf{Problema:}}\itshape
}{\end{beamercolorbox}}

\newcommand\halfcol{\column{.46\textwidth}}
\newcommand\onethirdcol{\column{.31\textwidth}}

\newcommand{\Oh}{\mathrm{O}}

% ?????????
\usepackage{subcaption}

\newcommand*\bolinha[1]{\; \tikz[inner sep=.25ex]\node[circle,draw]{#1}; \;}

%----------------------------------------------------------------------------------------
%	POSTER
%----------------------------------------------------------------------------------------

\title{Rainbow Version of Dirac's Theorem: An Algorithmic Approach}
\author{
  \begin{tabular}{l} % Define uma tabela com alinhamento à esquerda
    Nathan Luiz Bezerra Martins \hspace{200pt} Orientadora: Yoshiko Wakabayashi\\
    Willian Miura Mori
  \end{tabular}
}
\institute{\vspace{10pt}Departamento de Ciência da Computação,\\
Instituto de Matemática e Estatística, Universidade de São Paulo}

\begin{document}
\begin{frame}[fragile]\centering
  \vspace{-40pt}
  \begin{columns}[T]

    % ----------------------------------------------------------------------------------------
    % PRIMEIRA COLUNA
    % ----------------------------------------------------------------------------------------
    \onethirdcol
    \begin{alertblock}{Resumo}
      Dada uma coleção $G = {G_1, G_2, \ldots, G_n}$ de grafos de ordem $n$, definidos 
      sobre o mesmo conjunto de vértices e que satisfazem a condição de Dirac para cada 
      $G_i$, existe um $G$-transversal que forma um circuito hamiltoniano, também conhecido
      como Circuito Hamiltoniano Rainbow. Neste trabalho, desenvolvemos um 
      algoritmo eficiente que encontra um Circuito Hamiltoniano Rainbow. Fizemos implementações
      tanto em \texttt{C++} quanto em \texttt{Python} e realizamos testes de desempenho para comparar as duas versões.
      Utilizamos a biblioteca \texttt{manim} para fazer uma animação gráfica do algoritmo.
    \end{alertblock}

    \begin{block}{Conceitos básicos}
        \textbf{Definições principais:}
        \begin{itemize}
          \item Um grafo simples é um grafo não direcionado sem laços e sem arestas múltiplas.
          \item Um \textbf{ciclo hamiltoniano} de $G$ é um ciclo que visita cada vértice de $G$ exatamente uma vez.
          \item $\delta(G)$ é o grau mínimo de um vértice em $G$.
        \end{itemize}
        \vspace{0.5em}
      \textbf{Teorema de Dirac (1952):}
        Se um grafo simples $G$ com $n$ vértices tem grau mínimo $\delta(G) \geq \frac{n}{2}$, então $G$ contém um ciclo hamiltoniano.
    \end{block}

    \begin{block}{Versões Rainbow de problemas clássicos}
      % \textbf{Definição}

      % A versão rainbow de um problema na teoria dos grafos é uma variação que adiciona a restrição de "diversidade" ou "colorido" à solução desejada. Nesse contexto, o termo rainbow (arco-íris) refere-se a estruturas em um grafo que utilizam elementos provenientes de diferentes subconjuntos, ou arestas com diferentes rótulos ou cores, garantindo que não haja repetições.
      aaaa


    \end{block}

    % ----------------------------------------------------------------------------------------
    % SEGUNDA COLUNA
    % ----------------------------------------------------------------------------------------
    \onethirdcol

    \begin{exampleblock}


      As link-cut trees fornecem a seguinte interface:

      \definecolor{shadecolor}{rgb}{0.93, 0.80, 0.82} % pale pink
      \begin{shaded}
        \begin{itemize}
          \item[$\bullet$] \textbf{\texttt{make\_root(u)}}: enraíza no vértice $u$ a árvore que o contém
          \item[$\bullet$] \textbf{\texttt{link(u, v, w)}}: dado que os vértices $u$ e $v$ estão em árvores separadas, transforma $v$ em raiz de sua árvore e o liga como filho de $u$, colocando peso $w$ na nova aresta criada
          \item[$\bullet$] \textbf{\texttt{cut(u, v)}}: retira da floresta a aresta com pontas em $u$ e $v$, quebrando a árvore que continha estes vértices em duas novas árvores
          \item[$\bullet$] \textbf{\texttt{is\_connected(u, v)}}: retorna \texttt{verdadeiro} caso $u$ e $v$ pertençam à mesma árvore, \texttt{falso} caso contrário
          \item[$\bullet$] \textbf{\texttt{maximum\_edge(u, v)}}: retorna o peso máximo de uma aresta no caminho entre os vértices $u$ e $v$
        \end{itemize}
      \end{shaded}

      Todas essas operações consomem tempo $\Oh(\log n)$ amortizado, onde $n$ é o número de vértices na floresta.

    \end{exampleblock}

    \begin{block}{Union-Find retroativo}

      O union-find é uma estrutura de dados utilizada para manter uma \textbf{coleção de conjuntos disjuntos}, isto é, conjuntos que não se intersectam.

      \bigskip
      \begin{figure}[h!]
        \captionsetup{justification=centering}
        \centering
        \begin{subfigure}{\textwidth}
          \centering
          \begin{tikzpicture}[
              no/.style={shape=circle, minimum size=1cm},
              scale=1.5, transform shape
            ]
            \node[no] (a) at (4,0) {a};
            \node[no] (x) at (5,0) {};
            \node[no] (b) at (6,0) {b};
            \node[no] (y) at (7,0) {};
            \node[no] (c) at (8,0) {c};
            \node[no] (z) at (9,0) {};
            \node[no] (d) at (10,0) {d};

            \draw ($(x)$) ellipse ({1.7cm} and {.7cm});
            \draw ($(c)$) ellipse ({.7cm} and {.7cm});
            \draw ($(d)$) ellipse ({.7cm} and {.7cm});
          \end{tikzpicture}
          \bigskip
        \end{subfigure}
        \begin{subfigure}{\textwidth}
          \centering
          \begin{tikzpicture}[
              no/.style={shape=circle, minimum size=1cm},
              scale=1.5, transform shape
            ]
            \node[no] (a) at (4,0) {a};
            \node[no] (x) at (5,0) {};
            \node[no] (b) at (6,0) {b};
            \node[no] (y) at (7,0) {};
            \node[no] (c) at (8,0) {c};
            \node[no] (z) at (9,0) {};
            \node[no] (d) at (10,0) {d};

            \draw ($(x)$) ellipse ({1.7cm} and {.7cm});
            \draw ($(z)$) ellipse ({1.7cm} and {.7cm});
          \end{tikzpicture}
          \bigskip
        \end{subfigure}
        \begin{subfigure}{\textwidth}
          \centering
          \begin{tikzpicture}[
              no/.style={shape=circle, minimum size=1cm},
              scale=1.5, transform shape
            ]
            \node[no] (a) at (4,0) {a};
            \node[no] (x) at (5,0) {};
            \node[no] (b) at (6,0) {b};
            \node[no] (y) at (7,0) {};
            \node[no] (c) at (8,0) {c};
            \node[no] (z) at (9,0) {};
            \node[no] (d) at (10,0) {d};

            \draw ($(y)$) ellipse ({3.7cm} and {.7cm});
          \end{tikzpicture}
          \bigskip
        \end{subfigure}
        \begin{subfigure}{\textwidth}
          \centering
          \begin{tikzpicture}[
              no/.style={shape=circle, minimum size=1cm},
              scale=1.5, transform shape
            ]
            \node[no] (a) at (4,0) {a};
            \node[no] (x) at (5,0) {};
            \node[no] (b) at (6,0) {b};
            \node[no] (y) at (7,0) {};
            \node[no] (c) at (8,0) {c};
            \node[no] (z) at (9,0) {};
            \node[no] (d) at (10,0) {d};

            \draw ($(b)$) ellipse ({2.8cm} and {.7cm});
            \draw ($(d)$) ellipse ({.7cm} and {.7cm});
          \end{tikzpicture}
        \end{subfigure}
        \caption{Representação dos conjuntos com os elementos $\{a,b,c,d\}$ após a seguinte sequência de operações: \texttt{create\_union(a, b, 2)}, \texttt{create\_union(c, d, 3)}, \texttt{create\_union(b, c, 4)} e \texttt{delete\_union(3)}. Cada linha mostra o estado atual da coleção imediatamente após uma operação.}
        \label{fig:uf-sets}
      \end{figure}

      Na sua versão retroativa, implementamos as seguintes operações:

      \definecolor{shadecolor}{rgb}{0.93, 0.80, 0.82} % pale pink
      \begin{shaded}
        \begin{itemize}
          \item[$\bullet$] \textbf{\texttt{create\_union(a, b, t)}}: adiciona a união dos conjuntos que contém $a$ e $b$ no instante de tempo $t$
          \item[$\bullet$] \textbf{\texttt{same\_set(a, b, t)}}: consulta se dois elementos pertenciam ao mesmo conjunto no instante $t$
          \item[$\bullet$] \textbf{\texttt{delete\_union(t)}}: desfaz a união realizada em $t$
        \end{itemize}
      \end{shaded}


      Por exemplo, a Figura \ref{fig:uf-sets} mostra o estado de uma coleção de conjuntos disjuntos após quatro operações serem aplicadas. Antes da operação \texttt{delete\_union(3)}, as consultas \texttt{same\_set(a, b, 3)} e \texttt{same\_set(c, d, 3)} retornam \texttt{verdadeiro}. Por outro lado \texttt{same\_set(a, d, 3)} e \texttt{same\_set(c, d, 3)} retornam \texttt{falso} após a chamada da função \texttt{delete\_union(3)}.

      \definecolor{shadecolor}{rgb}{0.74, 0.83, 0.9} % pale blue
      \begin{shaded}
        \textbf{Ideia:} Fazer com que os elementos dos conjuntos sejam vértices na floresta mantida por uma link-cut tree, onde cada aresta representa uma operação de \texttt{union}. Assim, uma chamada \texttt{create\_union(a, b, 3)} cria uma aresta de valor $3$ entre os vértices $a$ e $b$. Da mesma forma, uma chamada \texttt{delete\_union(t)} simplesmente exclui a aresta criada no instante $t$. Para conferir se dois elementos $a$ e $b$, no instante de tempo $t$, estão em um mesmo conjunto, basta conferir se eles estão em uma mesma árvore e se o valor da maior aresta no caminho entre eles é menor ou igual a $t$, o que significa que todas as uniões já foram realizadas no instante consultado.
      \end{shaded}

    \end{block}




    % ----------------------------------------------------------------------------------------
    % TERCEIRA COLUNA
    % ----------------------------------------------------------------------------------------
    \onethirdcol

    \begin{block}{Floresta geradora mínima retroativa}

      Como passo inicial temos que introduzir a \textbf{floresta geradora mínima incremental}, uma estrutura que utiliza as link-cut trees para fornecer uma maneira eficiente de consulta acerca da floresta geradora mínima de um grafo que está sempre crescendo, isto é, que está sofrendo a inserção de novas arestas.

      \begin{figure}
        \centering
        \captionsetup{justification=centering}
        \begin{equation*}
          \overbracket{
            \overbracket{
              \overbracket{
                \overbracket{
                  \overbracket{
                    \bolinha{\,0\,}
                  }^{F_0}
                  \bolinha{\,1\,}
                  \bolinha{\,2\,}
                }^{F_1}
                \bolinha{\,3\,}
                \bolinha{\,4\,}
              }^{F_2}
              \bolinha{\,5\,}
              \bolinha{\,6\,}
            }^{F_3}
            \bolinha{\,7\,}
            \bolinha{\,8\,}
          }^{F_4}
        \end{equation*}
        \caption{Representação da lista de $8$ arestas inseridas. Neste caso, cada bloco tem tamanho $2$. Assim, por exemplo, a estrutura $F_3$ contém todas as arestas adicionadas desde o instante $1$ até o instante $6$.}
        \label{fig:sqrt-decomp-blocks-m16}
      \end{figure}

      A \textbf{floresta geradora mínima retroativa} tem a seguinte interface:
      \definecolor{shadecolor}{rgb}{0.93, 0.80, 0.82} % pale pink
      \begin{shaded}
        \begin{itemize}
          \item[$\bullet$] \textbf{\texttt{add\_edge(u, v, w, t)}}: adiciona no grafo, no instante $t$, uma aresta com pontas $u$ e $v$ e peso $w$
          \item[$\bullet$] \textbf{\texttt{get\_msf(t)}}: retorna a lista com todas as arestas que compõem uma floresta maximal de peso mínimo do grafo no instante $t$
          \item[$\bullet$] \textbf{\texttt{get\_msf\_weight(t)}}: retorna o custo de uma floresta maximal de peso mínimo no instante $t$
        \end{itemize}
      \end{shaded}

      \definecolor{shadecolor}{rgb}{0.74, 0.83, 0.9} % pale blue
      \begin{shaded}
        \textbf{Ideia:} Organizar cada operação retroativa de inserção numa lista ordenada pelo instante de tempo em que a aresta foi inserida. Em seguida, utilizar a técnica de \textbf{square-root decomposition} para dividir essa lista em $\sqrt{m}$ blocos, onde $m$ é o número total de operações na lista. Essa divisão --- ou como chamamos, reconstrução --- vai sendo refeita conforme novas operações de inserção vão sendo adicionadas, a fim de manter o tamanho dos blocos aproximadamente constante. Por último, é necessário distribuir as operações de cada bloco em diferentes florestas geradoras mínimas incrementais, fazendo com que uma consulta acerca do instante de tempo $t$ possa ser realizada de maneira eficiente por uma estrutura que contenha um grafo com um estado próximo ao instante $t$.
      \end{shaded}

      Por último, além da ideia inicial \citet{10.1093/comjnl/bxaa135} para a floresta geradora mínima retroativa, foi necessário adaptar a ideia apresentada por \citet{10.1145/1240233.1240236} para transformar estruturas parcialmente retroativas em estruturas totalmente retroativas. Em particular, realizamos uma melhoria na etapa de reconstrução da estrutura, permitindo que ela seja realizada em tempo $\Oh(m \log n)$, onde $n$ é o número de vértices na floresta. Adicionalmente, escrevemos um artigo descrevendo essa melhoria, visando a sua publicação em algum veículo da área teórica de ciência da computação.

    \end{block}

    \begin{block}{Informações e contato}
      Para mais informações, acesse a página do trabalho: \textcolor{jblue}{{\url{https://linux.ime.usp.br/~felipen/mac0499}}}

      \medskip
      Endereço para contato: \\ \textcolor{jblue}{{\url{felipe.castro.noronha@usp.br}}}
    \end{block}

    \begin{block}{Referências}
      \bibliography{bibliografia.bib}
    \end{block}


  \end{columns}
\end{frame}
\end{document}