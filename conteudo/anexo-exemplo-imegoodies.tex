%!TeX root=../tese.tex
%("dica" para o editor de texto: este arquivo é parte de um documento maior)
% para saber mais: https://tex.stackexchange.com/q/78101

\chapter{As packages \pkg{imegoodies} e \pkg{imelooks}}
\label{ann:imegoodlooks}

Este modelo inclui as \textit{packages} \pkg{imegoodies} e \pkg{imelooks},
que você pode querer usar em outros documentos \LaTeX.

\pkg{imegoodies} inclui um grande número de \textit{packages} que são
comumente usadas e bastante úteis. Em geral, você pode incluí-la em seus
documentos sem que isso cause problemas de compatibilidade. Se, no
entanto, algo não funcionar, você pode editar o arquivo para eliminar
a \textit{package} responsável pelo problema se ela não for necessária.
\pkg{imegoodies} ainda inclui vários comentários explicativos sobre as
\textit{packages} carregadas.

\pkg{imelooks} também inclui um grande número de \textit{packages}, mas
estas são relacionadas mais explicitamente à aparência do documento
(fontes, cores, margens etc.). Você também pode utilizá-la em outros
documentos se quiser se aproximar da aparência deste modelo. \pkg{imelooks}
reconhece diversos parâmetros que ativam/desativam aspectos específicos:

\begin{itemize}
  \item \cmd{fonts} carrega as fontes deste modelo (libertinus e
        sourcecodepro), além de outros pequenos ajustes relacionados.
        Esta opção é sempre ativada por padrão; para desativá-la, use
        \cmd{nofonts}

  \item \cmd{spacing} utiliza os espaçamentos definidos neste modelo (margens,
        espaço entre parágrafos, indentação da primeira linha do parágrafo
        etc.). Esta opção é sempre ativada por padrão; para desativá-la, use
        \cmd{nospacing}

  \item \cmd{captions} e \cmd{footnotes} fazem respectivamente as legendas
        (das figuras e tabelas) e as notas de rodapé de acordo com este modelo.
        Estas opções são sempre ativadas por padrão; para desativá-las, use
        \cmd{nocaptions} e \cmd{nofootnotes}

  \item \cmd{autohttp} acrescenta o prefixo \cmd{http://} a URLs criadas
        com \ltxcmd{url} que não incluam o \textit{schema}. Esta opção é
        sempre ativada por padrão; para desativá-la, use \cmd{noautohttp}

  \item \cmd{hidelinks}, \cmd{borderlinks} e \cmd{colorlinks} definem a
        aparência dos hiperlinks. \cmd{hidelinks} faz os hiperlinks sem
        nenhuma formatação especial; \cmd{borderlinks} faz os hiperlinks
        serem envidos por um quadrado colorido (apenas na tela; o quadrado
        não é impresso); \cmd{colorlinks} faz o texto dos hiperlinks ser
        colorido. A opção \cmd{colorlinks} é sempre ativada por padrão

  \item \cmd{biblatex} carrega a \textit{package} \cmd{biblatex} e os
        estilos bibliográficos deste modelo. Esta opção é sempre ativada
        por padrão; para desativá-la, use \cmd{nobiblatex}
  \item \cmd{raggedbib} faz a bibliografia (com \cmd{biblatex}) ser
        formatada com alinhamento à esquerda ao invés de justificado.
        Esta opção é sempre ativada por padrão, exceto quando o estilo
        bibliográfico é \cmd{plainnat-ime} (usado nas teses); para
        desativá-la, use \cmd{noraggedbib}; para ativá-la incondicionalmente,
        use \cmd{raggedbib}
  \item \cmd{bibstyle=?} selectiona um estilo bibliográfico específico.
        O estilo padrão é \cmd{numeric}, exceto em pôsteres e apresentações
        (\cmd{beamer-ime}) e \textit{reports} (\cmd{plainnat-ime})

  \item \cmd{listings} carrega a \textit{package} \cmd{listings} e diversas
        configurações relacionadas usadas neste modelo. Esta opção é
        sempre ativada por padrão; para desativá-la, use \cmd{nolistings}

  \item \cmd{greeny}, \cmd{bluey}, \cmd{sandy} ativam esquemas de cores
        diferentes para pôsteres e apresentações (o padrão é \cmd{bluey})

  \item \cmd{beamer} \textbf{des}ativa algumas \textit{packages} que
        são incompatíveis com a classe \cmd{beamer} (note que as opções
        \cmd{slides} e \cmd{presentation}, discutidas abaixo, já fazem isso)

  \item \cmd{presentation} (ou \cmd{slides}) e \cmd{poster} ativam as
        opções relevantes para, respectivamente, apresentações com
        \cmd{beamer} ou pôsteres com \cmd{tcolorbox}

  \item \cmd{report} ativa as opções relevantes para documentos com
        capítulos (cabeçalhos das páginas, características do sumário etc.)

  \item \cmd{thesis} ativa a opção \cmd{report} e também define o que é
        necessário para a geração da capa das teses de acordo com este modelo

  \item \cmd{resumoabstract} define os comandos \cmd{resumo} e \cmd{abstract}
        de acordo com este modelo. Esta opção é ativada por padrão com
        \cmd{report}; para desativá-la, use \cmd{noresumoabstract}

  \item \cmd{brazilian} verifica se a língua portuguesa está ativa no
        documento e, em caso negativo, gera um erro. Esta opção é
        ativada por padrão com a opção \cmd{thesis}; para desativá-la,
        use \cmd{nobrazilian}
\end{itemize}
