%!TeX root=../tese.tex
%("dica" para o editor de texto: este arquivo é parte de um documento maior)
% para saber mais: https://tex.stackexchange.com/q/78101

\chapter*{Introduction}   %%%% versao de 28 jan -- yw revisou 

The problem we address in this monograph is relatively recent, although its origin traces back to
classical concepts.  In 1978, Caccetta and Häggkvist
(\cite{nathanson2006caccettahaggkvistconjectureadditivenumber}) formulated the conjecture that every
simple digraph of order $n$ with minimum outdegree~$d$ has a directed cycle of length at most
$\lceil n/d \rceil$. This conjecture marked the beginning of investigations of short cycles in
digraphs with degree constraints.

Nearly four decades later, in 2017 Ron Aharoni, from the Department of Mathematics at Technion,
proposed a stronger version of this conjecture, known as a rainbow version
(\cite{AharonisConjecture}). His conjecture states that, given a graph~$G$ of order~$n$ with edges
colored in $n$ colors, if each color appears in at most $r$ edges, then $G$ has a rainbow cycle of
length at most~$\lceil n/r \rceil$.

In 2019, Felix Joos, from the University of Heidelberg, and Jaehoon Kim, from KAIST, proved the
existence of a rainbow Hamiltonian cycle in a graph $G$ that results from the union of $n$
edge-disjoint graphs $G_i$ ($1 \leq i \leq n$) defined on a same set of $n$ vertices, each $G_i$
monochromatically edge colored but collectively using $n$ distinct colors, and each one satisfying
Dirac's condition. A rainbow cycle is a cycle in which all edges are colored differently. This
result gave strong support to Aharoni's conjecture.

The proof presented by Joos and Kim uses simple and elegant (yet non-trivial) techniques that
enabled the development of an $O(n^3)$ algorithm in the number of vertices.  They also extended
their result proving the existence of a perfect rainbow matching.

More recently, in 2023 Liqing Gao and Jian Wang (\cite{gao_wang_2024}) proved the existence of a
rainbow Hamiltonian cycle in graphs that satisfy a condition stated in a theorem (also known as
Ore's theorem, but different from the one we mention in Chapter~....) using the ``shifting
operator'' tool. This technique, developed by Erdös, Ko, and Rado, is widely used in extremal set
theory and has led to significant advances in solving problems in this area. However, we will not
cover this work here.

This monograph is structured as follows. In Chapter~1, we present some basic definitions, then we
present the well-known Dirac's theorem obtained in 1952 (\cite{dirac1952}, and state the theorem
known as the \emph{rainbow version of Dirac's theorem}, the central topic of this monograph.  In
Chapter~2, we present a pseudocode and the details of the implementation of the algorithm based on
the work of Joos and Kim (\cite{Joos_2020}).  In Chapter~3, we present a statement which we call the
\emph{rainbow version of Ore's theorem}, which is analogous to the rainbow version of Dirac's
theorem, but is based on a (weaker) sufficient condition for a graph to be Hamiltonian proved by Ore
in 1960.  We conjecture that this statement on the existence of a rainbow Hamiltonian cycle in the
union graph $G$ is true. If so, this result would generalize the rainbow version of Dirac's
theorem. We show a partial result that we have obtained (so far) for this statement: that the union
graph $G$, of order $n$, has a rainbow Hamiltonian path and also  a rainbow cycle of length
$n-1$.  Finally, in Chapter~4, we make some final considerations about the work we have developed,
including a graphical visualizer for our implementation.

The source codes developed in this project are available in
\href{https://github.com/wmrmrx/TCC}{GitHub}. There is a code written in C++, using the Boost
library, and also a code in Python, as it supports an animation using the Graph-Tool framework.
