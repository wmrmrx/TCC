%!TeX root=../tese.tex
%("dica" para o editor de texto: este arquivo é parte de um documento maior)
% para saber mais: https://tex.stackexchange.com/q/78101

\chapter*{Introduction}

The problem we address in the monograph is relatively recent, although its origins trace back to classical concepts. 
In 1978, Caccetta and Häggkvist (\cite{nathanson2006caccettahaggkvistconjectureadditivenumber}) formulated the conjecture 
that every simple digraph of order $n$ with minimum outdegree $d$ 
has a directed cycle of size at most $\lceil n/d \rceil$. This conjecture marked the beginning of investigations 
of short cycles in digraphs with degree constraints.

Nearly four decades later, in 2017, Ron Aharoni, from the Department of Mathematics at Technion, proposed a stronger 
version of this conjecture known as a rainbow version (\cite{AharonisConjecture}). His conjecture states that, given a graph $G$ of order $n$ 
with edges colored in $n$ colors, where each color appears in at most $r$ edges, there is a rainbow cycle of size 
at most $\lceil n/r \rceil$.

In 2019, Felix Joos, from the University of Heidelberg, and Jaehoon Kim, from KAIST, 
proved the existence of a rainbow Hamiltonian cycle in a graph $G$ that results from the union of $n$ graphs
$G_i$ ($1 \leq i \leq n$) defined on the same set of $n$ vertices, each $G_i$ monochromatically edge colored
and collectively using $n$ distinct colors, each one satisfying Dirac's condition. A rainbows cycle is a cycle 
in which all edges are colored differently. This result gave strong support to Aharoni's conjecture.

Joos and Kim's work uses simple and elegant, yet non-trivial, techniques that enabled the development of an $O(n^3)$ 
algorithm in the number of vertices. Moreover, they extended their results to the existence of a perfect rainbow 
matching.

More recently, in 2023, Liqing Gao and Jian Wang (\cite{gao_wang_2024}), Chinese researchers, published a paper that extends the problem 
by proving the existence of a rainbow Hamiltonian cycle in graphs that satisfy a simpler Ore condition, using the 
"shifting operator" tool. This technique, developed by Erdös, Ko, and Rado, is widely used in extremal set theory 
and has led to significant advances in solving this problem. However, we will not cover this work in detail in 
this paper.

This work is structured as follows: In Chapter 1, prove Dirac's Theorem obtained in 1952 (\cite{dirac1952}) and address the Rainbow version of Dirac's 
Theorem. In chapter 2, we present a pseudocode and the details of the implementation of the algorithm based on the work of Joos and Kim (\cite{Joos_2020}). 
In chapter 3, we introduce the Rainbow version of Ore's Theorem, and work done in hope of proving such version.
Finally, in Chapter 4, we make some final considerations about the work we have developed, including a graphical visualizer for our implementation.

All the source code used in this project is available in \href{https://github.com/wmrmrx/TCC}{GitHub}. The code was written in C++, using the Boost 
library, and in Python, which supports an animation made using the Graph-Tool framework.




