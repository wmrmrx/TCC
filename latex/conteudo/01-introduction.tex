%!TeX root=../tese.tex
%("dica" para o editor de texto: este arquivo é parte de um documento maior)
% para saber mais: https://tex.stackexchange.com/q/78101

\chapter{Introduction}

O problema estudado neste artigo é relativamente recente, embora suas origens remontem a conceitos clássicos. 
Em 1978, Caccetta e Häggkvist formularam a conjectura de que todo dígrafo de ordem $n$ com grau 
mínimo $d$ possui um circuito direcionado de tamanho no máximo $\lceil n/d \rceil$. Essa 
conjectura marcou o início de investigações sobre circuitos em grafos com restrições de grau.

Quase quatro décadas depois, em 2017, Ron Aharoni, do Departamento de Matemática do Technion, 
propôs uma versão mais forte dessa conjectura utilizando a versão Rainbow. Sua conjectura
afirma que, dado um grafo $G$ de ordem $n$ com arestas coloridas em $n$ cores, onde cada cor 
aparece em no máximo $r$ arestas, existe um circuito rainbow de tamanho no máximo $\lceil n/r \rceil$.

Em 2019, Felix Joos, professor na Universidade de Heidelberg, e Jaehoon Kim, professor associado 
no KAIST, conseguiram provar a existência de um circuito hamiltoniano rainbow em uma coleção de 
grafos definidos sobre o mesmo conjunto de vértices, com arestas coloridas de forma distinta, 
que satisfazem a condição de Dirac. Essa prova deu um forte suporte à conjectura de Aharoni.

O trabalho de Joos e Kim utilizou técnicas simples e elegantes, mas não triviais, 
que permitiram o desenvolvimento de um algoritmo cúbico no número de vértices. Além disso, eles 
estenderam seus resultados para a existência de um matching perfeito rainbow.

Mais recentemente, em 2023, Liqing Gao e Jian Wang, pesquisadores chineses, publicaram um artigo 
que estende o problema ao provar a existência de um circuito hamiltoniano rainbow em grafos que 
satisfazem a condição de Ore, utilizando a ferramenta do "shifting operator". Esta técnica, 
desenvolvida por Erdös, Ko e Rado, é amplamente utilizada na teoria dos conjuntos extremais e 
permitiu um avanço significativo na resolução deste problema. Porém, não iremos abordar este
trabalho em detalhes neste artigo.

Este trabalho está estruturado da seguinte forma: no Capítulo 2, explicaremos e provaremos a 
condição de Dirac. No Capítulo 3, abordaremos a versão Rainbow dos algoritmos, essencial para a 
compreensão do algoritmo final. No Capítulo 4, implementaremos o algoritmo baseado no trabalho 
de Joos e Kim, apresentando pseudocódigos e provando a corretude do algoritmo. No Capítulo 5, 
incluiremos uma animação gráfica para ilustrar o funcionamento do algoritmo. Finalmente, no 
Capítulo 6, faremos as considerações finais sobre o trabalho desenvolvido.

Todo o código-fonte utilizado neste projeto está disponível em $nosso-repo$ 
O código foi escrito em C++, utilizando a biblioteca Boost, e em Python, no qual tem suporte para 
uma animação que foi feita usando o framework Graph-Tool.
