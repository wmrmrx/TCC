%!TeX root=../tese.tex
%("dica" para o editor de texto: este arquivo é parte de um documento maior)
% para saber mais: https://tex.stackexchange.com/q/78101

\chapter{Introduction}

The problem studied in this paper is relatively recent, although its origins trace back to classical concepts. 
In 1978, Caccetta and Häggkvist formulated the conjecture that every digraph of order $n$ with minimum degree $d$ 
has a directed cycle of size at most $\lceil n/d \rceil$. This conjecture marked the beginning of investigations 
into cycles in graphs with degree constraints.

Nearly four decades later, in 2017, Ron Aharoni, from the Department of Mathematics at Technion, proposed a stronger 
version of this conjecture using the Rainbow version. His conjecture states that, given a graph $G$ of order $n$ 
with edges colored in $n$ colors, where each color appears in at most $r$ edges, there is a rainbow cycle of size 
at most $\lceil n/r \rceil$.

In 2019, Felix Joos, professor at the University of Heidelberg, and Jaehoon Kim, associate professor at KAIST, 
managed to prove the existence of a rainbow Hamiltonian cycle in a collection of graphs defined on the same set 
of vertices, with edges distinctly colored, that satisfy the Dirac condition. This proof gave strong support to 
Aharoni's conjecture.

Joos and Kim's work used simple and elegant, yet non-trivial, techniques that enabled the development of a cubic 
algorithm in the number of vertices. Moreover, they extended their results to the existence of a perfect rainbow 
matching.

More recently, in 2023, Liqing Gao and Jian Wang, Chinese researchers, published a paper that extends the problem 
by proving the existence of a rainbow Hamiltonian cycle in graphs that satisfy the Ore condition, using the 
"shifting operator" tool. This technique, developed by Erdös, Ko, and Rado, is widely used in extremal set theory 
and has led to significant advances in solving this problem. However, we will not cover this work in detail in 
this paper.

This work is structured as follows: In Chapter 2, we will explain and prove Dirac's condition. In Chapter 3, we 
will address the Rainbow version of the algorithms, essential for understanding the final algorithm. In Chapter 4, 
we will implement the algorithm based on the work of Joos and Kim, presenting pseudocode and proving the correctness 
of the algorithm. In Chapter 5, we will include a graphical animation to illustrate the operation of the algorithm. 
Finally, in Chapter 6, we will make final considerations about the work developed.

All the source code used in this project is available in $our-repo$. The code was written in C++, using the Boost 
library, and in Python, which supports an animation made using the Graph-Tool framework.




