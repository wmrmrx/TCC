%%!TeX root=../tese.tex

\chapter{Conclusion}   %%% versao de 29 jan -- yw revisou %%%%%%%%

The main objective of this work was to understand well the proof of
the rainbow version of Dirac's theorem obtained by Joos and Kim in
2020, and present a detailed algorithmic proof that yields an
efficient algorithm to construct a rainbow Hamiltonian cycle
guaranteed by this theorem.  This objective was attained, as the
algorithm that we derived and implemented has asymptotically the best
possible time complexity: $O(n^3)$ for input graphs with $n$ vertices
and $O(n^3)$ edges.

The biggest challenge we faced in the implementation described in
Chapter~2 was in the construction of a rainbow Hamiltonian cycle from
a rainbow cycle of length~$n-1$, without increasing the time
complexity that was achieved up to that step.  We tested our
implementation on a large number of randomly generated graphs with
order in the range from 6 to 100, and observed that the practical
results confirmed our theoretical results.

In Chapter~3 we present a statement that we call the rainbow version
of Ore's theorem which is a generalization of the rainbow version of
Dirac's theorem. In this statement, each graph $G_i$ has to satisfy
Ore's condition: $d(u) + d(v) \geq n$ for every pair of
non-adjacent vertices $u$ and $v$ in $V(G_i)$.  We conjecture that
this statement is true. We obtained (so far) only a partial result on
this statement: that the union graph $G$ has a rainbow Hamiltonian
path and also a rainbow cycle of length $n-1$. This partial result is
interesting in its own, but we hope to extend this result and 
show -- in the near future -- that our conjecture is true.

We also developed an interactive visualization tool for our algorithm,
which is available in our Github repository at
\url{https://github.com/wmrmrx/TCC/tree/master/code/src_python}.

This monograph and our codes are all public. This way, we hope these
studies will bring contribution to the field.


% %%%%%%%%%%%%%%%%%%%%%%%%%%%
% The main objective of this work was to implement an efficient algorithm to find
% the Rainbow version of Dirac's Theorem based on the work done by Joos and Kim (\cite{Joos_2020}).
% The biggest challenge found in the implementation done in \autoref{chap:algorithmic} was 
% finding a $\mathcal{G}$-transversal of length $n$ from a $\mathcal{G}$-transversal of length $n - 1$.

% A battery of tests was made to test our implementation.

% We study the Rainbow version of Ore's Theorem in \autoref{chap:ore}, which is a generalization of
% the Rainbow version of Dirac's Theorem. Although we do a proof similar
% to the one done by Joos and Kim to prove we can find a 
% $ \mathcal{G} $-transversal that is a cycle of length $ n - 1 $, it remains
% to be seen if there is a proof of existence of a 
% $ \mathcal{G} $-transversal that is a Hamiltonian cycle.

% To facilitate understanding, we developed 
% an interactive visualization tool for our algorithm, which is available in 
% our Github repository at \url{https://github.com/wmrmrx/TCC/tree/master/code/src_python}.
% The work contributes to the field through its practical implementation.
