%!TeX root=../tese.tex

\chapter{Conclusion}

The main objective of this work was to implement an efficient algorithm to find
the Rainbow version of Dirac's Theorem based on the work done by Joos and Kim (\cite{Joos_2020}).
The biggest challenge found in the implementation done in \autoref{chap:algorithmic} was 
finding a $\mathcal{G}$-transversal of length $n$ from a $\mathcal{G}$-transversal of length $n - 1$.

A battery of tests was made to test our implementation.

We study the Rainbow version of Ore's Theorem in \autoref{chap:ore}, which is a generalization of
the Rainbow version of Dirac's Theorem. Although we do a proof similar
to the one done by Joos and Kim to prove we can find a 
$ \mathcal{G} $-transversal that is a cycle of length $ n - 1 $, it remains
to be seen if there is a proof of existence of a 
$ \mathcal{G} $-transversal that is a Hamiltonian cycle.

To facilitate understanding, we developed 
an interactive visualization tool for our algorithm, which is available in 
our Github repository at \url{https://github.com/wmrmrx/TCC/tree/master/code/src_python}.
The work contributes to the field through its practical implementation.
