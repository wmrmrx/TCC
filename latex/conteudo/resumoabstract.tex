%!TeX root=../tese.tex
%("dica" para o editor de texto: este arquivo é parte de um documento maior)
% para saber mais: https://tex.stackexchange.com/q/78101

% As palavras-chave são obrigatórias, em português e em inglês, e devem ser
% definidas antes do resumo/abstract. Acrescente quantas forem necessárias.
\palavraschave{Teorema de Dirac, Versão Rainbow, Circuito Hamiltoniano,  Implementação}



\keywords{Dirac Theorem, Rainbow version, Hamiltonian Cycle, Implementation}    %% Versao de 28 jan %%%%%%%%%%%%%%

\resumo{
Seja $n\geq 3$ e $G = G_1 \cup G_2 \cup \ldots \cup G_n$ um
grafo que é a união de $n$ grafos dois a dois aresta-disjuntos $G_i$
de ordem $n$, todos definidos sobre um mesmo conjunto de vértices,
cada qual com arestas monocromaticamente coloridas mas coletivamente
usando $n$ cores distintas.  Joos and Kim~(2020) provou que se cada
$G_i$ satisfaz a condição de Dirac (i.e. tem grau mínimo pelo menos
$n/2$), então $G$ tem um circuito Hamiltonian rainbow (um circuito em
que todas as arestas têm cores distintas).  Nesta monografia
apresentamos uma versão algoritmica dessa prova, e explicamos passo a
passo um algoritmo que constrói um circuito Hamiltoniano rainbow
em~$G$. Apresentamos um pseudocódigo para cada procedimento que é
descrito, detalhando as estruturas que são usadas e analisando sua
complexidade computacional. Mostramos que o algoritmo que
implementamos tem complexidade $O(n^3)$, assintoticamente a melhor
complexidade possível, pois o grafo de entrada~$G$ tem $O(n^3)$
arestas.  Adicionalmente, incluímos neste trabalho uma animação
gráfica que criamos para ilustrar o processo de construção de um
circuito Hamiltoniano rainbow no grafo~$G$.
}

\abstract{
Let $n\geq 3$ and $G = G_1 \cup G_2 \cup \ldots \cup
G_n$ be a graph that is the union of $n$ pairwise edge-disjoint graphs
$G_i$ of order $n$, all defined on a same vertex set, each one
monochromatically edge colored but collectively using $n$ distinct
colors. Joos and Kim~(2020) proved that if each $G_i$ satisfies
Dirac's condition (i.e. has minimum degree at least $n/2$), then $G$
has a rainbow Hamiltonian cycle (a cycle in which all edges have
disctinct colors).  In this monograph, we present an algorithmic
version of this proof, and explain step by step an algorithm that
builds a rainbow Hamiltonian cycle in~$G$. We present a pseudocode for
each procedure that is described, providing details on the structures
that are used and analysing its computational complexity.  We show
that the algorithm that we implemented has time complexity $O(n^3)$,
asymptotically the best possible, as the input graph~$G$ has $O(n^3)$
edges. Additionally, we include in this work a graphical animation
that we created to illustrate the process of building  a rainbow
Hamiltonian cycle in~$G$.
}
