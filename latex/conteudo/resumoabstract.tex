%!TeX root=../tese.tex
%("dica" para o editor de texto: este arquivo é parte de um documento maior)
% para saber mais: https://tex.stackexchange.com/q/78101

% As palavras-chave são obrigatórias, em português e em inglês, e devem ser
% definidas antes do resumo/abstract. Acrescente quantas forem necessárias.
\palavraschave{Teorema de Dirac, Versão Rainbow, Implementação}

\keywords{Dirac Theorem, Rainbow version, Implementation}

% O resumo é obrigatório, em português e inglês. Estes comandos também
% geram automaticamente a referência para o próprio documento, conforme
% as normas sugeridas da USP.
\resumo{
Dada uma coleção $G = {G_1, G_2, \ldots, G_n}$ de grafos de ordem $n$, definidos 
sobre o mesmo conjunto de vértices e que satisfazem a condição de Dirac para cada 
$G_i$, existe um $G$-transversal que forma um circuito Hamiltoniano, também conhecido
como Circuito Hamiltoniano Rainbow. Uma demonstração da existência desse circuito 
é apresentada em \cite{Joos_2020}. Neste trabalho, explicamos passo a passo um 
algoritmo que encontra um Circuito Hamiltoniano Rainbow, apresentando pseudocódigo
 para cada etapa, detalhando o funcionamento de cada uma delas e realizando uma 
 animação gráfica para ilustrar o processo.
}

\abstract{
Given a collection $G = {G_1, G_2, \ldots, G_n}$ of graphs of order $n$, defined 
over the same set of vertices and satisfying Dirac's condition for each $G_i$, 
there exists a $G$-transversal that forms a Hamiltonian circuit, also known as a 
Rainbow Hamiltonian Circuit. A proof of the existence of this circuit is presented 
in \cite{Joos_2020}. In this paper, we explain step by step an algorithm that finds 
a Rainbow Hamiltonian Circuit, providing pseudocode for each step, detailing the 
workings of each one, and creating a graphical animation to illustrate the process.
}
