%!TeX root=../tese.tex

\chapter{Rainbow version of Ore's Theorem}
\label{chap:ore}

In 1960, Ore~(\cite{Ore_1960}) proved the following result.

\bigskip

 \ni {\textbf{Theorem 1 (Ore, 1960).}}
  \textsl{If $G = (V, E)$ is a simple graph of order  $n\geq 3$  such that 
  $d_G(u) + d_G(v) \geq n$ for all pair of non-adjacent vertices $u, v \in V$,
  then $G$ contains a Hamiltonian cycle.}

\bigskip
  
In what follows, we shall refer to the above sufficient condition as \emph{Ore's condition}. 
It is immediate that Ore's theorem generalizes Dirac's theorem. So, a natural question that arises
is whether an analogous  rainbow version of Ore's Theorem also holds. We conjecture that
the answer is yes. For the moment, we are able to prove that the following (weaker) result holds.

\bigskip

\ni{\textbf{Theorem 2.}}
\textsl{Let $n\geq 3$ and $G = G_0 \cup G_1 \cup \ldots \cup G_{n-1}$ be a
graph that is the union of $n$ pairwise edge-disjoint simple graphs
$G_i$ of order $n$, all defined on a same vertex set, each one
monochromatically edge colored but collectively using $n$ distinct
colors. If each $G_i$ satisfies Ore's condition, then $G$ has a
rainbow path and a rainbow cycle of length $n-1$.}

\medskip

\begin{proof} %Proof of Theorem 2

As in the proof of the rainbow version of Dirac's theorem, we consider two cases.

\begin{itemize}
  
\item[]\textbf{Case (a):} Given a rainbow path in $G$ of length $\ell < n-1$, how to extend it to a rainbow path
of length $\ell +1 $ or to a rainbow cycle of length $\ell$ ou $\ell + 1$. \\

\item[]\textbf{Case (b):} Given a rainbow cycle in $G$ of length $\ell < n-1$, how to obtain a rainbow cycle of length $\ell + 1$ or a rainbow path of length $\ell + 1$.

\end{itemize}

\smallskip

We note that, it is immediate that $G$ has a rainbow path of length~$2$. Thus, starting with such a path,
and considering the two cases above, we are able to construct a rainbow path in $G$ of length $n-1$
and a rainbow  cycle of length $n-1$. 

From now on, whenever we refer to a path or a cycle, these are always
rainbow, so we omit stating this (but it should be understood). To
simplify notation, as here we do not have many indices for the
vertices, when  referring to an edge with endpoints $u$ and $v$, instead of 
writing  $\{u,v\}$, we may represent it as $uv$.  Also, to state that
an edge $uv$ has color $c$, we write $uv \in G_c$.

\bigskip


\ni \textbf{Case (a):  Path of length $\ell$}


Let $P = (x_0, e_0, \dots, x_{\ell-1}, e_{\ell-1}, x_{\ell})$ be a
path in $G$ of length~$\ell$. \\

\smallskip

\begin{itemize}
  
\item[]\textbf{Case (a1): \(\ell < n/2 \)} \\


Let \( c \) be a color that is not in the path $P$, and let  \(u \coloneqq x_0\) and \(v \coloneqq x_{\ell}\). 
If \(uv \in G_c \), then by adding the edge $vu$ to $P$ we obtain a cycle of length \(\ell+1 \).
% 
Else, since \( uv \not\in G_c \), by Ore's condition, we have that
\( d_{G_c}(u) + d_{G_c}(v) \geq n \), and hence there is a
vertex \( w \notin V(P)\) such that \( uw \in G_c \) or
\( vw \in G_c \). Indeed, if such a vertex $w$ does not exist, then
\( d_{G_c}(u) \leq \ell \) and \( d_{G_c}(v) \leq \ell \). But
in this case, we have that 
\( d_{G_c}(u) + d_{G_c}(v) \leq 2\ell < n \), a contradiction.
Given such a vertex \(w\), we can construct a path of length \(\ell+1 \)
by appending $w$ to the path~$P$ and using precisely one of the edges \(uw\) or \(vw\).\\ 


\item[] \textbf{Case (a2): \( n/2 \leq \ell < n-1 \)} \\ 


Let $P'$ be the path obtained from $P$ after removing its last vertex~$x_\ell$, i.e., $P' = (x_0, e_0, \dots, x_{l-1})$. There are two colors, say $c_0$ and $c_1$, that are not present
in~$P'$. We may assume, without loss of generality, that
\(G_0 \coloneqq G_{c_0}\) and \(G_1 \coloneqq G_{c_1}\).

Let
\(u \coloneqq x_0\) and \( v \coloneqq x_{\ell-1}\). 
% 
For a vertex $w\in V(G_i)$, let \( d^{\text{out}}_{G_i}(w)\) denote
the number of neighbors of \(w\) in $G_i$ that are not in the path $P$
(they are out of $P$).  Analogously, let \(d^{\text{in}}_{G_i}(w) \)
denote the number of neighbors of \(w\) in $G_i$ that are in the path
$P$.
%
By definition, \( d^{\text{in}}_{G_i}(w) + d^{\text{out}}_{G_i}(w) = d_{G_i}(w) \).

If \(uv \in G_0 \) or \( uv \in G_1 \), then we can extend $P'$ to a cycle of length~\( \ell \)
by adding $uv$ to $P'$.  Else, by Ore's condition, \( d_{G_0}(u) + d_{G_0}(v) \geq n \) and \( d_{G_1}(u) + d_{G_1}(v) \geq n \).
If there exists a vertex \(w\) not in~$P'$ such that both \(uw \in G_0 \) and \(vw \in G_1 \), 
then we can extend $P'$ to a  cycle of length \( \ell + 1 \) by adding $w$ and the edges $vw$ and $wv$.
Such an extension is also possible if \( uw \in G_1 \) and \( vw \in G_0 \).


So, let us assume that such a vertex \(w\) does not exist. In this case, we have that 
\( d^{\text{out}}_{G_0}(u) + d^{\text{out}}_{G_1}(v) \leq n - \ell \) and
\( d^{\text{out}}_{G_1}(u) + d^{\text{out}}_{G_0}(v) \leq n - \ell \). 
So, we have that \( d^{\text{in}}_{G_0}(u) + d^{\text{in}}_{G_0}(v) + 
d^{\text{in}}_{G_1}(u) + d^{\text{in}}_{G_1}(v)  \geq 2 n - 2 ( n - \ell  ) = 2 \ell \). We must have that 
either \( d^{\text{in}}_{G_0}(u) + d^{\text{in}}_{G_1}(v) \geq \ell \) or 
\( d^{\text{in}}_{G_1}(u) + d^{\text{in}}_{G_0}(v) \geq \ell \).
Suppose, without loss of generality, that \( d^{\text{in}}_{G_0}(u) + d^{\text{in}}_{G_1}(v) \geq \ell \).
By the Pigeonhole Principle, there exists \(i\) such that
\( ux_i \in G_0 \) and \(vx_{i-1} \in G_1 \). Thus, we can construct a
a cycle of length \( \ell \).


\end{itemize}


\ni \textbf{Case (b):  Cycle of length $\ell < n - 1$}

Let \( C = \{x_0, e_0, \dots, x_{\ell-1}, e_{\ell-1}, x_{\ell}\} \) be a cycle
of length \(\ell < n - 1 \), and let  $c_0$ and $c_1$ be two colors that
are not present in~$C$.  Define \(G_0 \coloneqq G_{c_0}\) and \(G_1 \coloneqq G_{c_1} \).

\smallskip

\begin{itemize}
  
\item[]\textbf{Case (b1): \(\ell < n/2 \)} \\

As \(G_0\) satisfies Ore's condition, \(G_0\) is connected.  Then,
there exists an edge \( uv \in G_0 \) such that \( u \in C \)
and \( v \not\in C \).  Let \(w\) be a vertex adjacent to \(u\) in
\(C\).  If \(vw \in G_1\), then we have a rainbow cycle of
length \(\ell+1\).  Else, since \( vw \not\in G_1 \), by Ore's
condition, \( d_{G_1}(v) + d_{G_1}(w) \geq n \), and therefore there exists 
a vertex \( z  \not\in C\) such that \(vz \in G_1 \) or
\( wz \in G_1 \). Indeed, if not, we would have \( d_{G_1}(v) \leq \ell \)
and \( d_{G_1}(w) \leq \ell - 1 \), and we could conclude that 
\( d_{G_1}(v) + d_{G_1}(w) \leq 2\ell - 1 < n \), a contradiction.
%
Given such a vertex \(z\), we can construct a path of length \(\ell+1 \) if
\( vz \in G_1 \) or \( wz \in G_1 \). \\ 
% yw tirou  isso (como z nao pertence a z, z \neq w):  or we can construct a cycle of length \(\ell+1 \) if
% \( \{w, z\} \in G_1 \) and \(z \neq w\).

  
\item[]\textbf{Case (b2): \(n/2 \leq  \ell < n-1 \)}\\

Suppose there exists two vertices \( u, v \not\in C \) such that
\( uv \in G_0 \) and \( uv \not\in G_1 \).  By Ore's
condition, \( d_{G_1}(u) + d_{G_1}(v) \geq n \), which implies the
existence of a vertex \( w \in C \) such that \( uw \in G_1 \) or
\( vw \in G_1 \) --- otherwise
\( d_{G_1}(u) + d_{G_1}(v) \leq 2 (n - \ell - 1) < n \). Thus, we can
build a rainbow path of length \( \ell+1 \).  Otherwise, we have that for
all vertices \( u, v \not\in C \), 
\[ uv \in G_0 \Longleftrightarrow uv \in G_1. \]
Since \(G_0\) and \(G_1\) are connected graphs, if there is a pair of vertices 
\(u, v \not\in C\) such that \( uv \in G_0 \), then we can obtain a 
path with endpoint in \(u\) or \(v\) and in a vertex in the cycle,
that has  length at least \( \ell + 1 \).

We now assume that there is no pair of vertices \(u, v \not \in C\)
such that \( uv \in G_0 \).  Consider \( u, v \not\in C \). By Ore's
condition, \( d_{G_0}(u) + d_{G_0}(v) \geq n \) and
\( d_{G_1}(u) + d_{G_1}(v) \geq n \).  We also know that every
neighbor of \(u\) and \(v\) in \(G_0\) (resp. \(G_1\)) is in the
cycle~$C$, and therefore,
\( d_{G_0}(u) + d_{G_0}(v) + d_{G_1}(u) + d_{G_1}(v) \geq 2n > 2\ell \).
We can assume without loss of generality that
\( d_{G_0}(u) + d_{G_1}(v) > \ell \).  This means that there exist
adjacent vertices \(w\) and \(z\) in \(C\) such that \( uw \in G_0 \)
and \( zv \in G_1 \).  Thus, we can remove the edge $wz$ from $C$ and
add to the resulting path $C-wz$ the edges $uw$ and $zv$ and obtain a
path of length \(\ell+1\).

\end{itemize}

\end{proof}


\bigskip

We hope that from the rainbow Hamiltonian path in $G$ guaranteed to
exist by Theorem 2, we may succeed proving that $G$ has a
rainbow Hamiltonian cycle. This result would give us a theorem that is
precisely \emph{the rainbow version of Ore's theorem}. Anyway, we
think Theorem~2 is an interesting result by its own right.
