%!TeX root=../tese.tex

\chapter{Ore Theorem Generalization}

\cite{Ore_1960} states that if a simple graph $G = (V, E)$ with $n$ vertices satisfies the condition 
$\deg_G(u) + \deg_G(v) \geq n$ for all $u, v \in V$ such that $\{u, v\} \not\in E$, then $G$ contains a hamiltonian cycle.
We will call such condition Ore's condition.

Ore's theorem is a generalization to Dirac's theorem, since if a graph satisfies Dirac's condition, it also satisfies Ore's condition.
We want to prove that if for a collection of graphs $G = \{G_1, G_2, \ldots, G_n\}$, they all satisfy Ore's condition, 
then there exists a rainbow hamiltonian cycle in \(G\).

\section{Flowchart}

\subsection{Path of Length $l$}

Suppose we have a path \( P = (x_1, e_1, \dots, x_{l}, e_{l}, x_{l + 1}) \) of length \( l \). 
Just like in Dirac's Rainbow version's proof case, 
we want to find either a path of length \( l+1 \) or a cycle of length \( l \) or \( l+1 \). 

\subsubsection{Case 1: \( l < \frac{n}{2} \)}

Let \( c \) be a color that is not in the path and \(u \coloneqq x_1\), \(v \coloneqq x_{l+1}\). 
If \( \{u, v\} \in G_c \), then we have a cycle of length \( l+1 \).
Else, since \( \{u, v\} \not\in G_c \), by Ore's condition, 
we have that \( \deg_{G_c}(u) + \deg_{G_c}(v) \geq n \), which implies there exists a 
vertex \( w \) such that \( \{u, w\} \in G_c \) or \( \{v, w\} \in G_c \), because otherwise
\( \deg_{G_c}(u) \leq l  \) and \( \deg_{G_c}(v) \leq l \), 
which would imply \( \deg_{G_c}(u) + \deg_{G_c}(v) \leq 2l < n \), a contradiction.
Given such \(w\), we can construct a path of length \( l+1 \) by appending \(w\) to the path.

\subsubsection{Case 2: \( l \geq \frac{n}{2} \)}

This case is analogous to Rainbow Dirac's equivalent case.

% TODO

\subsection{Cycle of Length $l < n - 1$}

Let \( C = \{x_1, e_1, \dots, x_{l}, e_{l}, x_{l + 1}\} \) be a cycle of length \( l < n - 1 \).

Let \( c_1, c_2 \) be colors that are not in the cycle. Define \(G_1 \coloneqq G_{c_1}, G_2 \coloneqq G_{c_2} \).

\subsubsection{Case 1: \( l < \frac{n}{2} \)}

From Ore's theorem, we know \(G_1\) is a connected graph, since it contains a hamiltonian cycle. 
Then, there exists a edge \( \{u, v\} \in E(G_1) \) such that \( u \in C \) and \( v \not\in C \). 
Let \(w\) be a vertex adjacent to \(u\) in \(C\). 
If \(\{v, w\} \in G_2\), then we have a rainbow cycle of length \( l+1 \). 
Else, since \( \{v, w\} \not\in G_2 \), by Ore's condition, \( \deg_{G_1}(v) + \deg_{G_1}(w) \geq n \),
which implies there exists a vertex \( z \) such that \( \{v, z\} \in G_2 \) or \( \{w, z\} \in G_2 \), because otherwise
\( \deg_{G_2}(v) \leq l \) and \( \deg_{G_2}(w) \leq l - 1 \), which implies 
\( \deg_{G_2}(v) + \deg_{G_2}(w) \leq 2l - 1 < n \), a contradiction.
Given such \(z\), we can construct a path of length \( l+1 \) if \( \{v, z\} \in G_2 \) or 
\( z = w \) and \( \{w, z\} \in G_2 \), or a cycle of length \( l+1 \) if \( \{w, z\} \in G_2 \) and \(z \neq w\).

\subsubsection{Case 2: \( l \geq \frac{n}{2} \)}

% TODO: Prove that symmetric difference of edges is empty
