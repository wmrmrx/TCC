%!TeX root=../tese.tex

\chapter{Rainbow version of Ore's Theorem}

\cite{Ore_1960} states that if a simple graph $G$ with $n$ vertices satisfies the condition
$\deg_G(u) + \deg_G(v) \geq n$ for all $u, v \in V$ such that ${u, v} \not\in E(G)$, then $G$ contains a hamiltonian cycle.
We will refer to this condition as Ore's condition.

Ore's theorem generalizes Dirac's theorem, since if a graph satisfies Dirac's condition, it also satisfies Ore's condition.
We want to prove that if for a collection of graphs $G = (G_0, G_1, \ldots, G_{n-1})$, they all satisfy Ore's condition,
then there exists a rainbow hamiltonian cycle in $G$.

\section{Flowchart}

\subsection{Path of Length $\ell$}

Suppose we have a path $ P = (x_0, e_0, \dots, x_{\ell-1}, e_{\ell-1}, x_{\ell}) $ of length $ \ell $.
As in the proof of the rainbow version of Dirac's theorem,
we want to find either a path of length $ \ell+1 $ or a cycle of length $ \ell $ or $ \ell+1 $.

We may assume that there always exists a path with $ \ell \geq 2 $, since every graph 
$ G_i $ satisfies $ \delta_{G_i}(v) \geq 2 $ for all $ v \in V(G_1) $ due to Ore's condition.

\subsubsection{Case 1: \( \ell < \frac{n}{2} \)}

Let \( c \) be a color that is not in the path and \(u \coloneqq x_0\), \(v \coloneqq x_{\ell}\). 
If \( \{u, v\} \in G_c \), then we have a cycle of length \( \ell+1 \).
Else, since \( \{u, v\} \not\in G_c \), by Ore's condition, 
we have that \( \deg_{G_c}(u) + \deg_{G_c}(v) \geq n \), which implies there exists a 
vertex \( w \) such that \( \{u, w\} \in G_c \) or \( \{v, w\} \in G_c \), because otherwise
\( \deg_{G_c}(u) \leq \ell  \) and \( \deg_{G_c}(v) \leq \ell \), 
which would imply \( \deg_{G_c}(u) + \deg_{G_c}(v) \leq 2l < n \), a contradiction.
Given such \(w\), we can construct a path of length \( \ell+1 \) by appending \(w\) to the path.

\subsubsection{Case 2: \( \ell \geq \frac{n}{2} \)}

This case is analogous to Rainbow Dirac's equivalent case.

Consider the path $P'$ as being the path $P$ with the last vertex removed, i.e., $P' = (x_0, e_0, \dots, x_{\ell-1}, e_{\ell-1})$.
There are two colors, $c_0$ and $c_1$, that are not in the path. Define 
\(G_0 \coloneqq G_{c_0}, G_1 \coloneqq G_{c_1}, u \coloneqq x_0, v \coloneqq x_{\ell-1}\), and 
\( \deg^{\text{out}}_{G_i}(w), \deg^{\text{in}}_{G_i}(w) \) as the number of neighbors of \(w \in V(G_i)\) that are not in the path
and the number of neighbors of \(v\) that are in the path, respectively.
By definition, \( \deg^{\text{in}}_{G_i}(w) +  \deg^{\text{out}}_{G_i}(w) = \deg_{G_i}(w) \).

If \( \{u, v\} \in G_0 \) or \( \{u, v\} \in G_1 \), then we have a cycle of length \( \ell \).
Else, by Ore's condition, \( \deg_{G_0}(u) + \deg_{G_0}(v) \geq n \) and \( \deg_{G_1}(u) + \deg_{G_1}(v) \geq n \).
If there exists a vertex \(w\) not in the path such that \( \{u, w\} \in G_0 \) and \( \{v, w\} \in G_1 \), 
then we have a cycle of length \( \ell + 1 \). The same is true if \( \{u, w\} \in G_1 \) and \( \{v, w\} \in G_0 \).

We can assume that such vertex \(w\) doesn't exists, implying 
\( \deg^{\text{out}}_{G_0}(u) + \deg^{\text{out}}_{G_1}(v) \leq n - \ell \) and
\( \deg^{\text{out}}_{G_1}(u) + \deg^{\text{out}}_{G_0}(v) \leq n - \ell \). 
So, we have that \( \deg^{\text{in}}_{G_0}(u) + \deg^{\text{in}}_{G_0}(v) + 
\deg^{\text{in}}_{G_1}(u) + \deg^{\text{in}}_{G_1}(v)  \geq 2 n - 2 ( n - \ell  ) = 2 \ell \). We must have that 
either \( \deg^{\text{in}}_{G_0}(u) + \deg^{\text{in}}_{G_1}(v) \geq \ell \) or 
\( \deg^{\text{in}}_{G_1}(u) + \deg^{\text{in}}_{G_0}(v) \geq \ell \).
Suppose, without loss of generality, that \( \deg^{\text{in}}_{G_0}(u) + \deg^{\text{in}}_{G_1}(v) \geq l \).
% TODO: detalhar mais
By the Pigeonhole Principle, there exists \(i\) such that 
\( \{u, x_i\} \in G_0 \) and \( \{v, x_{i-1}\} \in G_1 \).
Implying in the existence of a cycle of length \( \ell \).

\subsection{Cycle of Length $\ell < n - 1$}

Let \( C = \{x_0, e_0, \dots, x_{\ell-1}, e_{\ell-1}, x_{\ell}\} \) be a cycle of length \( \ell < n - 1 \).
There are two colors, $c_0$ and $c_1$, that are not in the cycle. Define \(G_0 \coloneqq G_{c_0}, G_1 \coloneqq G_{c_1} \).

\subsubsection{Case 1: \( \ell < \frac{n}{2} \)}

From Ore's theorem, we know \(G_0\) is a connected graph, since it contains a hamiltonian cycle. 
Then, there exists a edge \( \{u, v\} \in E(G_0) \) such that \( u \in C \) and \( v \not\in C \). 
Let \(w\) be a vertex adjacent to \(u\) in \(C\). 
If \(\{v, w\} \in G_1\), then we have a rainbow cycle of length \( \ell+1 \). 
Else, since \( \{v, w\} \not\in G_1 \), by Ore's condition, \( \deg_{G_1}(v) + \deg_{G_1}(w) \geq n \),
which implies there exists a vertex \( z \) such that \( \{v, z\} \in G_1 \) or \( \{w, z\} \in G_1 \), because otherwise
\( \deg_{G_1}(v) \leq \ell \) and \( \deg_{G_1}(w) \leq \ell - 1 \), implying 
\( \deg_{G_1}(v) + \deg_{G_1}(w) \leq 2\ell - 1 < n \), a contradiction.
Given such \(z\), we can construct a path of length \( \ell+1 \) if \( \{v, z\} \in G_1 \) or 
\( z = w \) and \( \{w, z\} \in E(G_1) \), or a cycle of length \( \ell+1 \) if \( \{w, z\} \in E(G_1) \) and \(z \neq w\).

\subsubsection{Case 2: \( \ell \geq \frac{n}{2} \)}

Suppose there exists two vertices \( u, v \not\in C \) such that \( \{u, v\} \in E(G_0) \) and \( \{u, v) \not\in E(G_1) \).
By Ore's condition, 
\( \deg_{G_1}(u) + \deg_{G_1}(v) \geq n \), 
implying there exists a vertex \( w \in C \) such that \( \{u, w\} \in E(G_1) \) or \( \{v, w\} \in E(G_1) \) 
--- otherwise \( \deg_{G_1}(u) + \deg_{G_1}(v) \leq 2 (n - \ell - 1) < n \).
With this, we can build a rainbow path of length \( \ell+1 \).
Otherwise, we have that for all \( u, v \not\in C \), 
\[ \{u, v\} \in E(G_0) \leftrightarrow \{u, v\} \in E(G_1). \]
Since \(G_0\) and \(G_1\) if there exists \(u, v \not\in C\) such that \( \{u, v\} \in E(G_0) \), 
then we have a path with endpoint in \(u\) or \(v\) and in a vertex in the cycle, creating a path of length at 
least \( \ell + 1 \).

We now assume that there is no such \(u, v \not \in C\) such that \( \{u, v\} \in E(G_0) \).
Consider \( u, v \not\in C \). By Ore's condition, 
\( \deg_{G_0}(u) + \deg_{G_0}(v) \geq n \) and \( \deg_{G_1}(u) + \deg_{G_1}(v) \geq n \). 
We also know that every neighbor of \(u\) and \(v\) in \(G_0, G_1\) is in the cycle, and
given \( \deg_{G_0}(u) + \deg_{G_0}(v) + \deg_{G_1}(u) + \deg_{G_1}(v) \geq 2n > 2\ell \), we 
can assume without loss of generality that \( \deg_{G_0}(u) + \deg_{G_1}(v) > \ell \).
This means that there exists adjacent vertices in \(C\) \(w, z\) such that
\( \{u, w\} \in E(G_0) \) and \( \{v, z\} \in E(G_1) \), creating a path of length \( \ell+1 \).