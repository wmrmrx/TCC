%!TeX root=../tese.tex
%("dica" para o editor de texto: este arquivo é parte de um documento maior)
% para saber mais: https://tex.stackexchange.com/q/78101

\chapter{Preliminary}

In this chapter, we present the concepts and theorems that are fundamental to understand the 
work developed in this monograph. We start by presenting and proving Dirac's theorem (from 1952) \cite{dirac1952} 
and then we formally present the rainbow version of this theorem, the main topic of this work.

We assume that the reader is familiar with some basic concepts of graph theory. We present some 
of them to estabilish the terminology and notation, which follow basically those used in Bondy and Murty \cite{bondy1976}.

\section{Definitions}

We denote by $G = (V, E)$ a simple graph $G$ with vertex set $V$ and edge set $E$.

A path $P$ of length $\ell$ is a sequence of vertices and edges $(v_0, e_0, v_1, e_1, \ldots, v_{\ell - 1}, e_{\ell - 1}, v_{\ell})$ such that
$v_i \in V$, $e_i = \{v_i, v_{i + 1}\} \in E$, and $v_i \neq v_j$ for $0 \leq i < j \leq \ell$.

A cycle $C$ of length $\ell$ in a graph $G = (V, E)$ is a sequence of vertices and edges $(v_0, e_0, v_1, e_1, \ldots, v_{\ell}, e_{\ell}, v_{\ell + 1})$ such that
$v_{0} = v_{\ell + 1}$, $v_i \in V$, $e_i = \{v_i, v_{i + 1}\} \in E$, and $v_i \neq v_j$ for $0 \leq i < j \leq \ell$.

When convenient, if $G$ is the name of a graph, we may refer to its vertex set and edge set as $V(G)$ and $E(G)$, respectively.

Given a simple graph $G = (V, E)$, a Hamiltonian cycle of $G$ is a cycle with length $|V|$.
Deciding whether a graph has a Hamiltonian cycle is a well-known NP-complete problem. 
However, there are many sufficient conditions that guarantee the existence of a Hamiltonian cycle in a graph. 
One of them, based on the minimum degree of the vertices, is given by the Dirac's theorem.

\section{Dirac's Theorem}

\textbf{Theorem 1 (Dirac, 1952)} $G = (V, E)$ with $n$ vertices satisfies the condition $\deg_G(v) \geq \frac{n}{2}$ for all $v \in V$, 
then $G$ contains a Hamiltonian cycle.

\begin{proof}
    Let $G = (V, E)$ with be a simple graph with n vertices that satisfies the condition of the theorem. Suppose, 
    by contradiction, that $G$ is not Hamiltonian. 

    Let $G' \coloneqq (V, E')$ be a graph that maximizes $|E'|$ such that $G'$ is not Hamiltonian and $E \subseteq E'$. 
    Clearly, $G'$ is not a complete graph, because otherwise it would be Hamiltonian. Consider an edge $e = \{x, y\} \not\in E'$.
    The graph $(V, E' + e)$ must contain a Hamiltonian cycle $C = (v_1, e_1, v_2, e_2, \ldots, v_n, e_{n}, v_1)$, where 
    $v_1 = x$, $v_n = y$ and $e_n = e$, because otherwise, it would contradict the maximality of $G'$.
    Since $G$ is a subgraph of $G'$, $\deg_G(v) \leq \deg_{G'}(v)$ for all $v \in V$.

    Let $I_1 \coloneqq \{i \in \{2, 3, \dots, n-2\} : \{x, v_{i+1}\} \in E'\}, 
    I_2 \coloneqq \{ i \in \{2, 3, \dots, n - 2\} : \{y, v_{i}\} \in E' \}$.
    We have that $|I_1| \geq \deg_{G'}(x) - 1$ and $|I_2| \geq \deg_{G'}(y) - 1$,
    which implies $|I_1| + |I_2| > n - 3$. By the Pigeonhole Principle, there exists $i \in I_1 \cup I_2$.

    That means that there exists a cycle 
    $(v_1, e_1, v_2, \dots, v_i, \{v_i, v_n\}, v_n, e_{n - 1}, v_{n-1} \dots , v_{i+1}, \{v_{i+1}, v_1\}, v_1)$, 
    which is Hamiltonian and is contained in $G$, a contradiction.

\end{proof}

\section{Rainbow Version of Dirac's Theorem}

Given a collection $\mathcal{G} = \{G_1, G_2, \ldots, G_n\}$ of graphs of order $n \geq 3$, each $G_i$ defined
on the same vertex set $V$ with $|V| = n$ and satisfying Dirac's condition for each $G_i$, there exists a 
$\mathcal{G}$-transversal that forms a Hamiltonian cycle, also known as a Rainbow Hamiltonian cycle.
Each graph $G_i$ can be viewed as if its edges were colored with color $i$.