%!TeX root=../tese.tex
%("dica" para o editor de texto: este arquivo é parte de um documento maior)
% para saber mais: https://tex.stackexchange.com/q/78101

\chapter{Preliminary}

In this chapter, we will present the concepts and theorems that are fundamental to understanding the 
work developed in this thesis. We will start by presenting and proving Dirac's theorem's original
statement and formally present the Rainbow version of the theorem.

\section{Definitions}

Given a simple graph $G = (V, E)$, where $V$ is the set of vertices and $E$ is the set of edges, we define $E(G)$ as
the set of edges of $G$, $V(G)$ as the set of vertices of $G$ and $\deg_G(v)$ as the degree of vertex $v$.

A path $P$ of length $l$ is a sequence of vertices and edges $(v_1, e_1, v_2, e_2, \ldots, v_{l}, e_{l}, v_{l + 1})$ such that
$v_i \in V(G)$, $e_i = \{v_i, v_{i + 1}\} \in E(G)$, and $v_i \neq v_j$ for $i \neq j$.

A cycle $C$ of length $l$ is a sequence of vertices and edges $(v_1, e_1, v_2, e_2, \ldots, v_{l}, e_{l}, v_{l + 1})$ such that
$v_{1} = v_{l + 1}$, $v_i \in V(G)$, $e_i = \{v_i, v_{i + 1}\} \in E(G)$, and $v_i \neq v_j$ for $i \neq j$.

\section{Hamiltonian Cycles}

Given a simple graph $G = (V, E)$, a hamiltonian cycle of $G$ is a cycle that contains $V$.
Finding whether a graph has a hamiltonian cycle is a well-known NP-complete problem. 
However, there are conditions that guarantee the existence of a hamiltonian cycle in a graph, one of them being Dirac's theorem.

\section{Dirac's Theorem}

\cite{dirac1952} states that if a simple graph $G = (V, E)$ with $n$ vertices satisfies the condition $\deg_G(v) \geq \frac{n}{2}$ for all $v \in V$, 
then $G$ contains a hamiltonian cycle.

\begin{proof}
    Let $G = (V, E)$ with be a simple graph with n vertices that satisfies the condition $\deg_G(v) \geq \frac{n}{2}$ for all $v \in V$.

    Suppose that $G$ is not hamiltonian. 

    Let $G' \coloneqq (V, E')$ be a graph that maximizes $|E'|$ such that $G'$ is not hamiltonian and $E \subseteq E'$. 
    $G'$ is not a complete graph, because otherwise it would be hamiltonian. Consider an edge $e' = \{x, y\} \not\in E'$.
    The graph $(V, E' + e)$ must contain a hamiltonian cycle $C = (v_1, e_1, v_2, e_2, \ldots, v_n, e_{n}, v_1)$, where 
    $v_1 = x$, $v_n = y$ and $e_n = e$, because otherwise, it would contradict the maximality of $G'$.
    Since $G$ is a subgraph of $G'$, $\deg_G(v) \leq \deg_{G'}(v)$ for all $v \in V$.

    Let $I_1 \coloneqq \{i \in \{2, 3, \dots, n-2\} : (x, v_{i+1}) \in E'\}, 
    I_2 \coloneqq \{ i \in \{2, 3, \dots, n - 2\} : (y, v_{i}) \in E' \}$.
    We have that $I_1 \geq deg_{G'}(x) - 1$ and $I_2 \geq deg_{G'}(y) - 1$,
    which implies $I_1 + I_2 > n - 3$. By the Pigeonhole Principle, there exists $i \in I_1 \cup I_2$.



\end{proof}

\section{Rainbow Version}

Given a collection of graphs  $G = \{G_1, G_2, \ldots, G_n\}$ defined on the same set of vertices
The rainbow version of Dirac's theorem states that if each graph $G_i$ satisfies Dirac's condition, 
then there is a hamiltonian cycle $C$ such that for all $i$, $C \cap G_i \neq \emptyset$.
