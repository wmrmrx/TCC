%!TeX root=../tese.tex
%("dica" para o editor de texto: este arquivo é parte de um documento maior)
% para saber mais: https://tex.stackexchange.com/q/78101

\chapter{Preliminary}

In this chapter, we present the concepts and theorems that are fundamental to understand the 
work developed in this monograph. We start by presenting and proving Dirac's theorem
(\cite{dirac1952}), 
and then we formally present the rainbow version of this theorem, the main topic of this work.

We assume that the reader is familiar with some basic concepts of graph theory. We present some 
of them to estabilish the terminology and notation, which follow those used in 
\cite{bondy1976}.

\section{Definitions}

We denote by $G = (V, E)$ a graph $G$ with vertex set $V$ and edge set $E$.

A path $P$ of size $\ell$ in a graph $G = (V, E)$ is a sequence of vertices and edges $(v_0, e_0, v_1, e_1, \ldots, v_{\ell - 1}, e_{\ell - 1}, v_{\ell})$ such that
$v_i \in V$, $e_i = \{v_i, v_{i + 1}\} \in E$, and $v_i \neq v_j$ for $0 \leq i < j \leq \ell$.

A cycle $C$ of size $\ell$ in a graph $G = (V, E)$ is a sequence of vertices and edges $(v_0, e_0, v_1, e_1, \ldots, v_{\ell}, e_{\ell}, v_{\ell + 1})$ such that
$v_{0} = v_{\ell + 1}$, $v_i \in V$, $e_i = \{v_i, v_{i + 1}\} \in E$, and $v_i \neq v_j$ for $0 \leq i < j \leq \ell$.

If $G$ is a graph with colors on the edges, a path (or cycle) in $G$ is called rainbow if all of its edges are colored differently.

When convenient, if $G$ is the name of a graph, we may refer to its vertex set and edge set as $V(G)$ and $E(G)$, respectively.

A Hamiltonian cycle in a graph $G$ of order $n \geq 3$ of length~$n$. The concept of Hamiltonian path is defined analogously (it contains all vertices of the graph).
A graph is Hamiltonian if it contains a Hamiltonian cycle.
Deciding whether a graph is Hamiltonian is a well-known NP-complete problem. 
However, there are many sufficient conditions that guarantee the existence of a Hamiltonian cycle in a graph. 
One of them, based on the minimum degree of the graph, is given by the Dirac's theorem.

\section{Dirac's Theorem}

\textbf{Theorem 1 (Dirac, 1952)} If a simple graph $G = (V, E)$ with $n \geq 3$ vertices satisfies the condition $\d_G(v)~\geq~\frac{n}{2}$ for all $v \in V$, 
then $G$ is Hamiltonian.

\begin{proof}
    Let $G = (V, E)$ with be a simple graph with $n \geq 3$ vertices that satisfies the condition of the theorem. Suppose, 
    by contradiction, that $G$ is not Hamiltonian. 

    Let $G' = (V, E')$ be a graph that maximizes $|E'|$ such that $G'$ is not Hamiltonian and $E \subseteq E'$. 
    Clearly, $G'$ is not a complete graph, because otherwise it would be Hamiltonian. Consider a pair $x, y \in V$ such that 
    $e = \{x, y\} \not\in E'$.
    The graph $(V, E' + e)$ must contain a Hamiltonian cycle $C = (v_1, e_1, v_2, e_2, \ldots, v_n, e_{n}, v_1)$, where 
    $v_1 = x$, $v_n = y$ and $e_n = e$, because otherwise, it would contradict the maximality of $G'$.
    Since $G$ is a subgraph of $G'$, $d_G(v) \leq d_{G'}(v)$ for all $v \in V$.

    Let $I_1 = \left\{i \in \{2, 3, \dots, n-2\} : \{x, v_{i+1}\} \in E'\right\}, 
    I_2 = \left\{ i \in \{2, 3, \dots, n - 2\} : \{y, v_{i}\} \in E' \right\}$.
    We have that $|I_1| \geq d_{G'}(x) - 1$ and $|I_2| \geq d_{G'}(y) - 1$,
    which implies $|I_1| + |I_2| > n - 3$. 
    Since $|I_1| + |I_2| = |I_1 \cup I_2| + |I_1 \cap I_2|$ and $|I_1 \cup I_2| \leq n - 3$, 
    there exists $i \in I_1 \cap I_2$.

    That means that there is a cycle 
    $(v_1, e_1, v_2, \dots, v_i, \{v_i, v_n\}, v_n, e_{n - 1}, v_{n-1} \dots , v_{i+1}, \{v_{i+1}, v_1\}, v_1)$, 
    which is Hamiltonian and is contained in $G'$, a contradiction. Thus, $G'$ does not exist,
    and therefore $G$ must be Hamiltonian.

\end{proof}

\section{Rainbow version of Dirac's Theorem}

Define rainbow path (or cycle) as a path (or cycle) in which all edges have distinct colors.

Let $n\geq 3$ and $G = G_1 \cup G_2 \cup \ldots \cup
G_n$ be a graph that is the union of $n$ pairwise edge-disjoint graphs
$G_i$ of order $n$, all defined on a same vertex set, each one
monochromatically edge colored but collectively using $n$ distinct
colors. If each $G_i$ satisfies Dirac's condition (i.e. has minimum 
degree at least $n/2$), then $G$ has a rainbow Hamiltonian cycle.
