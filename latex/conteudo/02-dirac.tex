%!TeX root=../tese.tex
%("dica" para o editor de texto: este arquivo é parte de um documento maior)
% para saber mais: https://tex.stackexchange.com/q/78101

\chapter{Preliminary}

In this chapter, we will present the concepts and theorems that are fundamental to understanding the 
work developed in this thesis. We will start by presenting and proving Dirac's theorem on its original
version and also formally present the Rainbow version of the theorem.

\section{Definitions}

Given a graph $G = (V, E)$, where $V$ is the set of vertices and $E$ is the set of edges, we define $E(G)$ as
the set of edges of $G$, $V(G)$ as the set of vertices of $G$ and $deg(v)$ as the degree of vertex $v$.

For a colection $G_1, G_2, \ldots, G_n$ of graphs defined on the same set of vertices, we define
$deg(v, G_i)$ as the degree of vertex $v$ in graph $G_i$.

A path $P$ of length $l$ is a sequence of vertices and edges $v_1, e_1, v_2, e_2, \ldots, v_{l}, e_{l}, v_{l + 1}$ such that
$v_i \in V(G)$, $e_i = \{v_i, v_{i + 1}\} \in E(G)$, and $v_i \neq v_j$ for $i \neq j$.

A circuit $C$ of length $l$ is also a sequence of vertices and edges $v_1, e_1, v_2, e_2, \ldots, v_{l}, e_{l}, v_{1}$ such that
$v_{1} = v{l + 1}$, $v_i \in V(G)$, $e_i = \{v_i, v_{i \cong l + 1}\} \in E(G)$, and $v_i \neq v_j$ for $i \neq j$.

\section{Hamiltonian Cycles}

Given a graph $G = (V, E)$, a hamiltonian cycle of $G$ is a cycle that visits every vertex of $G$ exactly once.
Finding whether a graph has a hamiltonian cycle is a well-known NP-complete problem. 
However, there are conditions that guarantee the existence of a hamiltonian cycle in a graph, one of them being Dirac's theorem.

\section{Dirac's Theorem}

\section{Rainbow Version}

Given a collection of graphs  $G = \{G_1, G_2, \ldots, G_n\}$ defined on the same set of vertices
The rainbow version of Dirac's theorem states that if each graph $G_i$ satisfies Dirac's condition, 
then there is a hamiltonian cycle $C$ such that for all $i$, $C \cap G_i \neq \emptyset$.
