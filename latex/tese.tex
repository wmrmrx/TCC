% Arquivo LaTeX de exemplo de dissertação/tese a ser apresentada à CPG do IME-USP
%
% Criação: Jesús P. Mena-Chalco
% Revisão: Fabio Kon e Paulo Feofiloff
% Adaptação para UTF8, biblatex e outras melhorias: Nelson Lago
%
% Except where otherwise indicated, these files are distributed under
% the MIT Licence. The example text, which includes the tutorial and
% examples as well as the explanatory comments in the source, are
% available under the Creative Commons Attribution International
% Licence, v4.0 (CC-BY 4.0) - https://creativecommons.org/licenses/by/4.0/


%%%%%%%%%%%%%%%%%%%%%%%%%%%%%%%%%%%%%%%%%%%%%%%%%%%%%%%%%%%%%%%%%%%%%%%%%%%%%%%%
%%%%%%%%%%%%%%%%%%%%%%%%%%%%%%% PREÂMBULO LaTeX %%%%%%%%%%%%%%%%%%%%%%%%%%%%%%%%
%%%%%%%%%%%%%%%%%%%%%%%%%%%%%%%%%%%%%%%%%%%%%%%%%%%%%%%%%%%%%%%%%%%%%%%%%%%%%%%%

\documentclass[a4paper,12pt,twoside,brazilian,english]{book}
\usepackage{imegoodies}
\usepackage[thesis]{imelooks}

%\nocolorlinks % para impressão em P&B

\graphicspath{{figuras/},{fig/},{logos/},{img/},{images/},{imagens/}}

% Comandos rápidos para mudar de língua:
% \en -> muda para o inglês
% \br -> muda para o português
% \texten{blah} -> o texto "blah" é em inglês
% \textbr{blah} -> o texto "blah" é em português
\babeltags{br = brazilian, en = english}

%%%%%%%%%%%%%%%%%%%%%%%%%%%%%%%%%%%%%%%%%%%%%%%%%%%%%%%%%%%%%%%%%%%%%%%%%%%%%%%%
%%%%%%%%%%%%%%%%%%%%%%%%%%%%%%%%%% METADADOS %%%%%%%%%%%%%%%%%%%%%%%%%%%%%%%%%%%
%%%%%%%%%%%%%%%%%%%%%%%%%%%%%%%%%%%%%%%%%%%%%%%%%%%%%%%%%%%%%%%%%%%%%%%%%%%%%%%%

% O arquivo com os dados bibliográficos para biblatex; você pode usar
% este comando mais de uma vez para acrescentar múltiplos arquivos
\addbibresource{bibliografia.bib}

% Este comando permite acrescentar itens à lista de referências sem incluir
% uma referência de fato no texto (pode ser usado em qualquer lugar do texto)
%\nocite{bronevetsky02,schmidt03:MSc, FSF:GNU-GPL, CORBA:spec, MenaChalco08}
% Com este comando, todos os itens do arquivo .bib são incluídos na lista
% de referências
%\nocite{*}

% É possível definir como determinadas palavras podem (ou não) ser
% hifenizadas; no entanto, a hifenização automática geralmente funciona bem
\babelhyphenation{documentclass latexmk soft-ware clsguide} % todas as línguas
\babelhyphenation[brazilian]{Fu-la-no}
\babelhyphenation[english]{what-ever}

% Estes comandos definem o título e autoria do trabalho e devem sempre ser
% definidos, pois além de serem utilizados para criar a capa, também são
% armazenados nos metadados do PDF. O subtítulo é opcional.
\title{Rainbow title}[Rainbow subtitle]
\translatedtitle{Arco-iris titulo}[Arco-iris subtítulo]

\author[fem]{Nathan Luiz, Willian Mori}

\def\profa{Prof\kern.02em.\kern-.07emª\kern.07em}
\def\dra{Dr\kern-.04em.\kern-.11emª\kern.07em}

% Para TCCs, este comando define o supervisor
\orientador[fem]{\profa{} \dra{} Yoshiko Wakabayashi (TODO: em ingles)}

\banca{
  \profa{} \dra{} Fulana de Tal (orientadora) -- IME-USP [sem ponto final],
  % Em inglês, não há o "ª"
  %Prof. Dr. Fulana de Tal (advisor) -- IME-USP [sem ponto final],
  Prof. Dr. Ciclano de Tal -- IME-USP [sem ponto final],
  \profa{} \dra{} Convidada de Tal -- IMPA [sem ponto final],
}

\tipotese{
  tcc,
  programa={Computer Science},
}

\defesa{
  local={São Paulo},
  data=1999-11-11, % YYYY-MM-DD
}

\direitos{CC-BY}

% Para gerar a ficha catalográfica, acesse https://fc.ime.usp.br/,
% preencha o formulário e escolha a opção "Gerar Código LaTeX".
% Basta copiar e colar o resultado aqui.
\fichacatalografica{}


%%%%%%%%%%%%%%%%%%%%%%%%%%%%%%%%%%%%%%%%%%%%%%%%%%%%%%%%%%%%%%%%%%%%%%%%%%%%%%%%
%%%%%%%%%%%%%%%%%%%%%%% AQUI COMEÇA O CONTEÚDO DE FATO %%%%%%%%%%%%%%%%%%%%%%%%%
%%%%%%%%%%%%%%%%%%%%%%%%%%%%%%%%%%%%%%%%%%%%%%%%%%%%%%%%%%%%%%%%%%%%%%%%%%%%%%%%

\begin{document}

%%%%%%%%%%%%%%%%%%%%%%%%%%% CAPA E PÁGINAS INICIAIS %%%%%%%%%%%%%%%%%%%%%%%%%%%%

% Aqui começa o conteúdo inicial que aparece antes do capítulo 1, ou seja,
% página de rosto, resumo, sumário etc. O comando frontmatter faz números
% de página aparecem em algarismos romanos ao invés de arábicos e
% desabilita a contagem de capítulos.
\frontmatter

\pagestyle{plain}

\onehalfspacing % Espaçamento 1,5 na capa e páginas iniciais

\maketitle % capa e folha de rosto

%%%%%%%%%%%%%%%% DEDICATÓRIA, AGRADECIMENTOS, RESUMO/ABSTRACT %%%%%%%%%%%%%%%%%%

\begin{dedicatoria}
Esta seção é opcional e fica numa página separada; ela pode ser usada para
uma dedicatória ou epígrafe.
\end{dedicatoria}

% Reinicia o contador de páginas (a próxima página recebe o número "i") para
% que a página da dedicatória não seja contada.
\pagenumbering{roman}

%!TeX root=../tese.tex
%("dica" para o editor de texto: este arquivo é parte de um documento maior)
% para saber mais: https://tex.stackexchange.com/q/78101

% As palavras-chave são obrigatórias, em português e em inglês, e devem ser
% definidas antes do resumo/abstract. Acrescente quantas forem necessárias.
\palavraschave{Palavra-chave1, Palavra-chave2, Palavra-chave3}

\keywords{Keyword1,Keyword2,Keyword3}

% O resumo é obrigatório, em português e inglês. Estes comandos também
% geram automaticamente a referência para o próprio documento, conforme
% as normas sugeridas da USP.
\resumo{
Dada uma coleção $G = {G_1, G_2, \ldots, G_n}$ de grafos de ordem $n$, definidos 
sobre o mesmo conjunto de vértices e que satisfazem a condição de Dirac para cada 
$G_i$, existe um $G$-transversal que forma um circuito hamiltoniano, também conhecido
como Circuito Hamiltoniano Rainbow. Uma demonstração da existência desse circuito 
é apresentada em \cite{paper}. Neste trabalho, explicamos passo a passo um 
algoritmo que encontra um Circuito Hamiltoniano Rainbow, apresentando pseudocódigo
 para cada etapa, detalhando o funcionamento de cada uma delas e realizando uma 
 animação gráfica para ilustrar o processo.
}

\abstract{
Given a collection $G = {G_1, G_2, \ldots, G_n}$ of graphs of order $n$, defined 
over the same set of vertices and satisfying Dirac's condition for each $G_i$, 
there exists a $G$-transversal that forms a Hamiltonian circuit, also known as a 
Rainbow Hamiltonian Circuit. A proof of the existence of this circuit is presented 
in \cite{paper}. In this paper, we explain step by step an algorithm that finds 
a Rainbow Hamiltonian Circuit, providing pseudocode for each step, detailing the 
workings of each one, and creating a graphical animation to illustrate the process.
}


%%%%%%%%%%%%%%%%%%%%%%%%%%%%%%%% CAPÍTULOS %%%%%%%%%%%%%%%%%%%%%%%%%%%%%%%%%%%%%

\mainmatter
\pagestyle{mainmatter}
\singlespacing

\pagestyle{mainmatter}
%!TeX root=../tese.tex
%("dica" para o editor de texto: este arquivo é parte de um documento maior)
% para saber mais: https://tex.stackexchange.com/q/78101

\chapter{Introduction}

O problema estudado neste artigo é relativamente recente, embora suas origens remontem a conceitos clássicos. 
Em 1978, Caccetta e Häggkvist formularam a conjectura de que todo dígrafo de ordem $n$ com grau 
mínimo $d$ possui um circuito direcionado de tamanho no máximo $\lceil n/d \rceil$. Essa 
conjectura marcou o início de investigações sobre circuitos em grafos com restrições de grau.

Quase quatro décadas depois, em 2017, Ron Aharoni, do Departamento de Matemática do Technion, 
propôs uma versão mais forte dessa conjectura utilizando a versão Rainbow. Sua conjectura
afirma que, dado um grafo $G$ de ordem $n$ com arestas coloridas em $n$ cores, onde cada cor 
aparece em no máximo $r$ arestas, existe um circuito rainbow de tamanho no máximo $\lceil n/r \rceil$.

Em 2019, Felix Joos, professor na Universidade de Heidelberg, e Jaehoon Kim, professor associado 
no KAIST, conseguiram provar a existência de um circuito hamiltoniano rainbow em uma coleção de 
grafos definidos sobre o mesmo conjunto de vértices, com arestas coloridas de forma distinta, 
que satisfazem a condição de Dirac. Essa prova deu um forte suporte à conjectura de Aharoni.

O trabalho de Joos e Kim utilizou técnicas simples e elegantes, mas não triviais, 
que permitiram o desenvolvimento de um algoritmo cúbico no número de vértices. Além disso, eles 
estenderam seus resultados para a existência de um matching perfeito rainbow.

Mais recentemente, em 2023, Liqing Gao e Jian Wang, pesquisadores chineses, publicaram um artigo 
que estende o problema ao provar a existência de um circuito hamiltoniano rainbow em grafos que 
satisfazem a condição de Ore, utilizando a ferramenta do "shifting operator". Esta técnica, 
desenvolvida por Erdös, Ko e Rado, é amplamente utilizada na teoria dos conjuntos extremais e 
permitiu um avanço significativo na resolução deste problema. Porém, não iremos abordar este
trabalho em detalhes neste artigo.

Este trabalho está estruturado da seguinte forma: no Capítulo 2, explicaremos e provaremos a 
condição de Dirac. No Capítulo 3, abordaremos a versão Rainbow dos algoritmos, essencial para a 
compreensão do algoritmo final. No Capítulo 4, implementaremos o algoritmo baseado no trabalho 
de Joos e Kim, apresentando pseudocódigos e provando a corretude do algoritmo. No Capítulo 5, 
incluiremos uma animação gráfica para ilustrar o funcionamento do algoritmo. Finalmente, no 
Capítulo 6, faremos as considerações finais sobre o trabalho desenvolvido.

Todo o código-fonte utilizado neste projeto está disponível em $nosso-repo$ 
O código foi escrito em C++, utilizando a biblioteca Boost, e em Python, no qual tem suporte para 
uma animação que foi feita usando o framework Graph-Tool.


\pagestyle{mainmatter}
%!TeX root=../tese.tex
%("dica" para o editor de texto: este arquivo é parte de um documento maior)
% para saber mais: https://tex.stackexchange.com/q/78101

\chapter{Preliminares}

\section{Teorema de Dirac}
\subsection{Hamiltonian Cycles}

Given a graph $G = (V, E)$, a hamiltonian cycle of $G$ is a cycle that visits every vertex of $G$ exactly once.
Finding whether a graph has a hamiltonian cycle is a well-known NP-complete problem. 
However, there are conditions that guarantee the existence of a hamiltonian cycle in a graph, one of them being Dirac's theorem.

\subsubsection{Dirac's Theorem}

\section{Versão Rainbow}

A versão Rainbow do Teorema de Dirac foi proposta por Ron Aharoni em 2017. 


%%%% Anexos %%%%

\cleardoublepage

\pagestyle{appendix} % repete o anterior, caso você não use apêndices

\annex

% \addappheadtotoc acrescenta a palavra "Anexo" ao sumário; se
% só há anexos, sem apêndices, provavelmente não é necessário.

%\addappheadtotoc
%
%\input{conteudo/anexo-exemplo-imegoodies}
%\par
%\input{conteudo/anexo-exemplo-pseudocodigo}
%\par


%%%%%%%%%%%%%%% SEÇÕES FINAIS (BIBLIOGRAFIA E ÍNDICE REMISSIVO) %%%%%%%%%%%%%%%%

% O comando backmatter desabilita a numeração de capítulos.
\backmatter

\pagestyle{backmatter}

% Espaço adicional no sumário antes das referências / índice remissivo
\addtocontents{toc}{\vspace{2\baselineskip plus .5\baselineskip minus .5\baselineskip}}

% A bibliografia é obrigatória

\printbibliography[
  title=\refname\label{sec:bib}, % "Referências", recomendado pela ABNT
  %title=\bibname\label{sec:bib}, % "Bibliografia"
  heading=bibintoc, % Inclui a bibliografia no sumário
]

\printindex % imprime o índice remissivo no documento (opcional)

\end{document}
