% Arquivo LaTeX de exemplo de dissertação/tese a ser apresentada à CPG do IME-USP
%
% Criação: Jesús P. Mena-Chalco
% Revisão: Fabio Kon e Paulo Feofiloff
% Adaptação para UTF8, biblatex e outras melhorias: Nelson Lago
%
% Except where otherwise indicated, these files are distributed under
% the MIT Licence. The example text, which includes the tutorial and
% examples as well as the explanatory comments in the source, are
% available under the Creative Commons Attribution International
% Licence, v4.0 (CC-BY 4.0) - https://creativecommons.org/licenses/by/4.0/


%%%%%%%%%%%%%%%%%%%%%%%%%%%%%%%%%%%%%%%%%%%%%%%%%%%%%%%%%%%%%%%%%%%%%%%%%%%%%%%%
%%%%%%%%%%%%%%%%%%%%%%%%%%%%%%% PREÂMBULO LaTeX %%%%%%%%%%%%%%%%%%%%%%%%%%%%%%%%
%%%%%%%%%%%%%%%%%%%%%%%%%%%%%%%%%%%%%%%%%%%%%%%%%%%%%%%%%%%%%%%%%%%%%%%%%%%%%%%%

\documentclass[a4paper,12pt,twoside,brazilian,english]{book}
\usepackage{imegoodies}
\usepackage[thesis]{imelooks}

%\nocolorlinks % para impressão em P&B

\graphicspath{{figuras/},{fig/},{logos/},{img/},{images/},{imagens/}}

% Comandos rápidos para mudar de língua:
% \en -> muda para o inglês
% \br -> muda para o português
% \texten{blah} -> o texto "blah" é em inglês
% \textbr{blah} -> o texto "blah" é em português
\babeltags{br = brazilian, en = english}

%%%%%%%%%%%%%%%%%%%%%%%%%%%%%%%%%%%%%%%%%%%%%%%%%%%%%%%%%%%%%%%%%%%%%%%%%%%%%%%%
%%%%%%%%%%%%%%%%%%%%%%%%%%%%%%%%%% METADADOS %%%%%%%%%%%%%%%%%%%%%%%%%%%%%%%%%%%
%%%%%%%%%%%%%%%%%%%%%%%%%%%%%%%%%%%%%%%%%%%%%%%%%%%%%%%%%%%%%%%%%%%%%%%%%%%%%%%%

% O arquivo com os dados bibliográficos para biblatex; você pode usar
% este comando mais de uma vez para acrescentar múltiplos arquivos
\addbibresource{bibliografia.bib}

% Este comando permite acrescentar itens à lista de referências sem incluir
% uma referência de fato no texto (pode ser usado em qualquer lugar do texto)
%\nocite{bronevetsky02,schmidt03:MSc, FSF:GNU-GPL, CORBA:spec, MenaChalco08}
% Com este comando, todos os itens do arquivo .bib são incluídos na lista
% de referências
%\nocite{*}

% É possível definir como determinadas palavras podem (ou não) ser
% hifenizadas; no entanto, a hifenização automática geralmente funciona bem
\babelhyphenation{documentclass latexmk soft-ware clsguide} % todas as línguas
\babelhyphenation[brazilian]{Fu-la-no}
\babelhyphenation[english]{what-ever}

% Estes comandos definem o título e autoria do trabalho e devem sempre ser
% definidos, pois além de serem utilizados para criar a capa, também são
% armazenados nos metadados do PDF. O subtítulo é opcional.
\title{Rainbow title}[Rainbow subtitle]
\translatedtitle{Arco-iris titulo}[Arco-iris subtítulo]

\author[fem]{Nathan Luiz, Willian Mori}

\def\profa{Prof\kern.02em.\kern-.07emª\kern.07em}
\def\dra{Dr\kern-.04em.\kern-.11emª\kern.07em}

% Para TCCs, este comando define o supervisor
\orientador[fem]{\profa{} \dra{} Yoshiko Wakabayashi (TODO: em ingles)}

\banca{
  \profa{} \dra{} Fulana de Tal (orientadora) -- IME-USP [sem ponto final],
  % Em inglês, não há o "ª"
  %Prof. Dr. Fulana de Tal (advisor) -- IME-USP [sem ponto final],
  Prof. Dr. Ciclano de Tal -- IME-USP [sem ponto final],
  \profa{} \dra{} Convidada de Tal -- IMPA [sem ponto final],
}

% A página de rosto da versão para depósito (ou seja, a versão final
% antes da defesa) deve ser diferente da página de rosto da versão
% definitiva (ou seja, a versão final após a incorporação das sugestões
% da banca).
\tipotese{
  %mestrado,
  %doutorado,
  tcc,
  %definitiva, % É a versão para defesa ou a versão definitiva?
  %quali, % É qualificação?
  programa={Computer Science},
}

\defesa{
  local={São Paulo},
  data=1999-11-11, % YYYY-MM-DD
}

% A licença do seu trabalho. Use CC-BY, CC-BY-NC, CC-BY-ND, CC-BY-SA,
% CC-BY-NC-SA ou CC-BY-NC-ND para escolher a licença Creative Commons
% correspondente (o sistema insere automaticamente o texto da licença).
% Se quiser estabelecer regras diferentes para o uso de seu trabalho,
% converse com seu orientador e coloque o texto da licença aqui, mas
% observe que apenas TCCs sob alguma licença Creative Commons serão
% acrescentados ao BDTA. Se você tem alguma intenção de publicar o
% trabalho comercialmente no futuro, sugerimos a licença CC-BY-NC-ND.
%
%\direitos{CC-BY-NC-ND}
%
%\direitos{Autorizo a reprodução e divulgação total ou parcial deste
%          trabalho, por qualquer meio convencional ou eletrônico,
%          para fins de estudo e pesquisa, desde que citada a fonte.}
%
%\direitos{I authorize the complete or partial reproduction and disclosure
%          of this work by any conventional or electronic means for study
%          and research purposes, provided that the source is acknowledged.}
%
\direitos{CC-BY}

% Para gerar a ficha catalográfica, acesse https://fc.ime.usp.br/,
% preencha o formulário e escolha a opção "Gerar Código LaTeX".
% Basta copiar e colar o resultado aqui.
\fichacatalografica{}


%%%%%%%%%%%%%%%%%%%%%%%%%%%%%%%%%%%%%%%%%%%%%%%%%%%%%%%%%%%%%%%%%%%%%%%%%%%%%%%%
%%%%%%%%%%%%%%%%%%%%%%% AQUI COMEÇA O CONTEÚDO DE FATO %%%%%%%%%%%%%%%%%%%%%%%%%
%%%%%%%%%%%%%%%%%%%%%%%%%%%%%%%%%%%%%%%%%%%%%%%%%%%%%%%%%%%%%%%%%%%%%%%%%%%%%%%%

\begin{document}

%%%%%%%%%%%%%%%%%%%%%%%%%%% CAPA E PÁGINAS INICIAIS %%%%%%%%%%%%%%%%%%%%%%%%%%%%

% Aqui começa o conteúdo inicial que aparece antes do capítulo 1, ou seja,
% página de rosto, resumo, sumário etc. O comando frontmatter faz números
% de página aparecem em algarismos romanos ao invés de arábicos e
% desabilita a contagem de capítulos.
\frontmatter

\pagestyle{plain}

\onehalfspacing % Espaçamento 1,5 na capa e páginas iniciais

\maketitle % capa e folha de rosto

%%%%%%%%%%%%%%%% DEDICATÓRIA, AGRADECIMENTOS, RESUMO/ABSTRACT %%%%%%%%%%%%%%%%%%

\begin{dedicatoria}
Esta seção é opcional e fica numa página separada; ela pode ser usada para
uma dedicatória ou epígrafe.
\end{dedicatoria}

% Reinicia o contador de páginas (a próxima página recebe o número "i") para
% que a página da dedicatória não seja contada.
\pagenumbering{roman}

% Agradecimentos:
% Se o candidato não quer fazer agradecimentos, deve simplesmente eliminar
% esta página. A epígrafe, obviamente, é opcional; é possível colocar
% epígrafes em todos os capítulos. O comando "\chapter*" faz esta seção
% não ser incluída no sumário.
\chapter*{Agradecimentos}
\epigrafe{Do. Or do not. There is no try.}{Mestre Yoda}

Texto texto texto texto texto texto texto texto texto texto texto texto texto
texto texto texto texto texto texto texto texto texto texto texto texto texto
texto texto texto texto texto texto texto texto texto texto texto texto texto
texto texto texto texto. Texto opcional.

%!TeX root=../tese.tex
%("dica" para o editor de texto: este arquivo é parte de um documento maior)
% para saber mais: https://tex.stackexchange.com/q/78101

% As palavras-chave são obrigatórias, em português e em inglês, e devem ser
% definidas antes do resumo/abstract. Acrescente quantas forem necessárias.
\palavraschave{Palavra-chave1, Palavra-chave2, Palavra-chave3}

\keywords{Keyword1,Keyword2,Keyword3}

% O resumo é obrigatório, em português e inglês. Estes comandos também
% geram automaticamente a referência para o próprio documento, conforme
% as normas sugeridas da USP.
\resumo{
Dada uma coleção $G = {G_1, G_2, \ldots, G_n}$ de grafos de ordem $n$, definidos 
sobre o mesmo conjunto de vértices e que satisfazem a condição de Dirac para cada 
$G_i$, existe um $G$-transversal que forma um circuito hamiltoniano, também conhecido
como Circuito Hamiltoniano Rainbow. Uma demonstração da existência desse circuito 
é apresentada em \cite{paper}. Neste trabalho, explicamos passo a passo um 
algoritmo que encontra um Circuito Hamiltoniano Rainbow, apresentando pseudocódigo
 para cada etapa, detalhando o funcionamento de cada uma delas e realizando uma 
 animação gráfica para ilustrar o processo.
}

\abstract{
Given a collection $G = {G_1, G_2, \ldots, G_n}$ of graphs of order $n$, defined 
over the same set of vertices and satisfying Dirac's condition for each $G_i$, 
there exists a $G$-transversal that forms a Hamiltonian circuit, also known as a 
Rainbow Hamiltonian Circuit. A proof of the existence of this circuit is presented 
in \cite{paper}. In this paper, we explain step by step an algorithm that finds 
a Rainbow Hamiltonian Circuit, providing pseudocode for each step, detailing the 
workings of each one, and creating a graphical animation to illustrate the process.
}



%%%%%%%%%%%%%%%%%%%%%%%%%%% LISTAS DE FIGURAS ETC. %%%%%%%%%%%%%%%%%%%%%%%%%%%%%

% Como as listas que se seguem podem não incluir uma quebra de página
% obrigatória, inserimos uma quebra manualmente aqui.
\cleardoublepage

% Todas as listas são opcionais; Usando "\chapter*" elas não são incluídas
% no sumário. As listas geradas automaticamente também não são incluídas por
% conta das opções "notlot" e "notlof" que usamos para a package tocbibind.

% Normalmente, "\chapter*" faz o novo capítulo iniciar em uma nova página, e as
% listas geradas automaticamente também por padrão ficam em páginas separadas.
% Como cada uma destas listas é muito curta, não faz muito sentido fazer isso
% aqui, então usamos este comando para desabilitar essas quebras de página.
% Se você preferir, comente as linhas com esse comando e des-comente as linhas
% sem ele para criar as listas em páginas separadas. Observe que você também
% pode inserir quebras de página manualmente (com \clearpage, veja o exemplo
% mais abaixo).
\newcommand\disablenewpage[1]{{\let\clearpage\par\let\cleardoublepage\par #1}}

% Nestas listas, é melhor usar "raggedbottom" (veja basics.tex). Colocamos
% a opção correspondente e as listas dentro de um grupo para ativar
% raggedbottom apenas temporariamente.
\bgroup
\raggedbottom

%%%%% Listas criadas manualmente

%\chapter*{Lista de abreviaturas}
\disablenewpage{\chapter*{Lista de abreviaturas}}

\begin{tabular}{rl}
   CFT & Transformada contínua de Fourier (\emph{Continuous Fourier Transform})\\
   DFT & Transformada discreta de Fourier (\emph{Discrete Fourier Transform})\\
  EIIP & Potencial de interação elétron-íon (\emph{Electron-Ion Interaction Potentials})\\
  STFT & Transformada de Fourier de tempo reduzido (\emph{Short-Time Fourier Transform})\\
  ABNT & Associação Brasileira de Normas Técnicas\\
   URL & Localizador Uniforme de Recursos (\emph{Uniform Resource Locator})\\
   IME & Instituto de Matemática e Estatística\\
   USP & Universidade de São Paulo
\end{tabular}

%\chapter*{Lista de símbolos}
\disablenewpage{\chapter*{Lista de símbolos}}

\begin{tabular}{rl}
  $\omega$ & Frequência angular\\
    $\psi$ & Função de análise \emph{wavelet}\\
    $\Psi$ & Transformada de Fourier de $\psi$\\
\end{tabular}

% Quebra de página manual
\clearpage

%%%%% Listas criadas automaticamente

% Você pode escolher se quer ou não permitir a quebra de página
%\listoffigures
\disablenewpage{\listoffigures}

% Você pode escolher se quer ou não permitir a quebra de página
%\listoftables
\disablenewpage{\listoftables}

% Esta lista é criada "automaticamente" pela package float quando
% definimos o novo tipo de float "program" (em utils.tex)
% Você pode escolher se quer ou não permitir a quebra de página
%\listof{program}{\programlistname}
\disablenewpage{\listof{program}{\programlistname}}

% Sumário (obrigatório)
\tableofcontents

\egroup % Final de "raggedbottom"

% Referências indiretas ("x", veja "y") para o índice remissivo (opcionais,
% pois o índice é opcional). É comum colocar esses itens no final do documento,
% junto com o comando \printindex, mas em alguns casos isso torna necessário
% executar texindy (ou makeindex) mais de uma vez, então colocar aqui é melhor.
\index{Inglês|see{Língua estrangeira}}
\index{Figuras|see{Floats}}
\index{Tabelas|see{Floats}}
\index{Código-fonte|see{Floats}}
\index{Subcaptions|see{Subfiguras}}
\index{Sublegendas|see{Subfiguras}}
\index{Equações|see{Modo matemático}}
\index{Fórmulas|see{Modo matemático}}
\index{Rodapé, notas|see{Notas de rodapé}}
\index{Captions|see{Legendas}}
\index{Versão original|see{Tese/Dissertação, versões}}
\index{Versão corrigida|see{Tese/Dissertação, versões}}
\index{Palavras estrangeiras|see{Língua estrangeira}}
\index{Floats!Algoritmo|see{Floats, ordem}}


%%%%%%%%%%%%%%%%%%%%%%%%%%%%%%%% CAPÍTULOS %%%%%%%%%%%%%%%%%%%%%%%%%%%%%%%%%%%%%

\mainmatter
\pagestyle{mainmatter}
\singlespacing

\pagestyle{unnumberedchapter}
%!TeX root=../tese.tex
%("dica" para o editor de texto: este arquivo é parte de um documento maior)
% para saber mais: https://tex.stackexchange.com/q/78101

\chapter**{Introduction}
\label{cap:introduction}

\enlargethispage{.5\baselineskip}

In this 

\section**{Implementation}

In this 


\pagestyle{mainmatter}
\input{conteudo/01-exemplo-normas-ime}
\input{conteudo/02-exemplo-usando-o-modelo}
\input{conteudo/03-exemplo-latex}
\input{conteudo/04-exemplo-tutorial}
\input{conteudo/05-exemplo-exemplos}


%%%%%%%%%%%%%%%%%%%%%%%%%%%% APÊNDICES E ANEXOS %%%%%%%%%%%%%%%%%%%%%%%%%%%%%%%%

% Um apêndice é algum conteúdo adicional de sua autoria que faz parte e
% colabora com a ideia geral do texto mas que, por alguma razão, não precisa
% fazer parte da sequência do discurso; por exemplo, a demonstração de um
% teorema intermediário, as perguntas usadas em uma pesquisa qualitativa etc.
%
% Um anexo é um documento que não faz parte da tese (em geral, nem é de sua
% autoria) mas é relevante para o conteúdo; por exemplo, a especificação do
% padrão técnico ou a legislação que o trabalho discute, um artigo de jornal
% apresentando a percepção do público sobre o tema da tese etc.
%
% Os comandos appendix e annex reiniciam a numeração de capítulos e passam
% a numerá-los com letras. "annex" não faz parte de nenhuma classe padrão,
% foi criado para este modelo. Se o trabalho não tiver apêndices ou anexos,
% remova estas linhas.
%
% Diferentemente de \mainmatter, \backmatter etc., \appendix e \annex não
% forçam o início de uma nova página. Em geral isso não é importante, pois
% o comando seguinte costuma ser "\chapter", mas pode causar problemas com
% a formatação dos cabeçalhos. Assim, vamos forçar uma nova página antes
% de cada um deles.

%%%% Apêndices %%%%

\cleardoublepage

\pagestyle{appendix}

\appendix

% \addappheadtotoc acrescenta a palavra "Apêndice" ao sumário; se
% só há apêndices, sem anexos, provavelmente não é necessário.
\addappheadtotoc

\input{conteudo/apendice-exemplo-faq}
\par

%%%% Anexos %%%%

\cleardoublepage

\pagestyle{appendix} % repete o anterior, caso você não use apêndices

\annex

% \addappheadtotoc acrescenta a palavra "Anexo" ao sumário; se
% só há anexos, sem apêndices, provavelmente não é necessário.
\addappheadtotoc

\input{conteudo/anexo-exemplo-imegoodies}
\par
\input{conteudo/anexo-exemplo-pseudocodigo}
\par


%%%%%%%%%%%%%%% SEÇÕES FINAIS (BIBLIOGRAFIA E ÍNDICE REMISSIVO) %%%%%%%%%%%%%%%%

% O comando backmatter desabilita a numeração de capítulos.
\backmatter

\pagestyle{backmatter}

% Espaço adicional no sumário antes das referências / índice remissivo
\addtocontents{toc}{\vspace{2\baselineskip plus .5\baselineskip minus .5\baselineskip}}

% A bibliografia é obrigatória

\printbibliography[
  title=\refname\label{sec:bib}, % "Referências", recomendado pela ABNT
  %title=\bibname\label{sec:bib}, % "Bibliografia"
  heading=bibintoc, % Inclui a bibliografia no sumário
]

\printindex % imprime o índice remissivo no documento (opcional)

\end{document}
